\documentclass[a4paper,12pt]{article}
\usepackage{fancyhdr}
\usepackage{fancyheadings}
\usepackage[ngerman,german]{babel}
\usepackage{german}
\usepackage[utf8]{inputenc}
\usepackage[active]{srcltx}
\usepackage{algorithm}
\usepackage[noend]{algorithmic}
\usepackage{amsmath}
\usepackage{amssymb}
\usepackage{amsthm}
\usepackage{bbm}
\usepackage{enumerate}
\usepackage{graphicx}
\usepackage{ifthen}
\usepackage{listings}
\usepackage{struktex}
\usepackage{hyperref}
\usepackage[T1]{fontenc}
\usepackage{amsfonts}

\newcommand{\Fach}{Wahrscheinlichkeitstheorie}
\newcommand{\Name}{Yannick Brenning, Jean Röther}
\newcommand{\Seminargruppe}{F}
\newcommand{\Matrikelnummer}{3732848, 3796826}
\newcommand{\Semester}{WiSe 22/23}
\newcommand{\Uebungsblatt}{4}

\setlength{\parindent}{0em}
\topmargin -1.0cm
\oddsidemargin 0cm
\evensidemargin 0cm
\setlength{\textheight}{9.2in}
\setlength{\textwidth}{6.0in}

\newcommand{\Aufgabe}[1]{
        {
        \vspace*{0.5cm}
        \textbf{HA #1}
        \vspace*{0.2cm}
    }
}

\newcommand{\PAufgabe}[1]{
        {
        \vspace*{0.5cm}
        \textbf{PA #1}
        \vspace*{0.2cm}
    }
}

\hypersetup{
    pdftitle = {\Fach{}: Übungsblatt \Uebungsblatt{}},
    pdfauthor = {\Name},
    pdfborder = {0 0 0}
}

\lstset{
    language=java,
    basicstyle=\footnotesize\tt,
    showtabs=false,
    tabsize=2,
    captionpos=b,
    breaklines=true,
    extendedchars=true,
    showstringspaces=false,
    flexiblecolumns=true,
}

\title{Übungsblatt \Uebungsblatt{}}
\author{\Name{}}

\begin{document}
    \thispagestyle{fancy}
    \lhead{\Fach{} \\ \small \Name{} - \Matrikelnummer{}}
    \rhead{\Semester{} \\  Übungsgruppe \Seminargruppe{}}
    \vspace*{0.2cm}
    \begin{center}
        \LARGE \textbf{Übungsblatt \Uebungsblatt{}}
    \end{center}
    \vspace*{0.2cm}

    \PAufgabe{2}

    Im folgenden beschreibe $ S_j, j \in \{0, \dots, 5\} $ das Ereignis, dass aus Urne $ U_j $ eine schwarze Kugel gezogen wird:

    \bigskip

    $ \mathbb{P}(S_0) = 0, \mathbb{P}(S_1) = \frac{1}{5}, \mathbb{P}(S_2) = \frac{2}{5}, \mathbb{P}(S_3) = \frac{3}{5}, \mathbb{P}(S_4) = \frac{4}{5}, \mathbb{P}(S_5) = 1 $

    \bigskip

    $ A = \{\text{Aus einer zuf"alligen Urne wird eine schwarze Kugel gezogen}\} $

    $ \displaystyle \mathbb{P}(A) = \sum\limits_{j = 1}^{5} \frac{1}{6} \cdot \mathbb{P}(S_j) = \frac{0 + 1 + 2 + 3 + 4 + 5}{6 \cdot 5} = \frac{15}{35} = \frac{3}{7} $

    \Aufgabe{2}
    \begin{enumerate}[(a)]
        \item 
        $ \displaystyle \mathbb{P}(G_1) = \frac{5}{12} $

        \item
        $ \displaystyle \mathbb{P}(O_1) = \frac{3}{12} = \frac{1}{4} $

        \item
        $ \displaystyle \mathbb{P}(G_2 | R_1) = \frac{\mathbb{P}(G_2 \cap R_1)}{\mathbb{P}(R_1)} = \frac{\frac{5}{11} \cdot \frac{1}{12}}{\frac{1}{12}}  = \frac{5}{132} \cdot \frac{12}{1} = \frac{60}{132} = \frac{5}{11} $

        \item
        $ \displaystyle \mathbb{P}(O_2 | O_1) = \frac{\mathbb{P}(O_2 \cap O_1)}{\mathbb{P}(O_1)} = \frac{\frac{2}{11} \cdot \frac{3}{12}}{\frac{3}{12}} = \frac{6}{132} \cdot \frac{12}{3} = \frac{72}{396} = \frac{2}{11} $

        \item
        $ \displaystyle \mathbb{P}(W_2 \cap O_1) = \frac{3}{11} \cdot \frac{3}{12} = \frac{9}{132} $

        \item
        $ \displaystyle \mathbb{P}(W_2) = \frac{5}{12} \cdot \frac{3}{11} + \frac{3}{12} \cdot \frac{3}{11} + \frac{3}{12} \cdot \frac{2}{11} + \frac{1}{12} \cdot \frac{3}{11} = \frac{33}{132} $

        \item
        $ \displaystyle \mathbb{P}(O_1 | W_2) = \frac{\mathbb{P}(O_1 \cap W_2)}{\mathbb{P}(W_2)} = \frac{\frac{3}{12} \cdot \frac{3}{11}}{\frac{33}{132}} = \frac{3}{11} $
    \end{enumerate}

    \Aufgabe{3}

    $ \displaystyle \mathbb{P}(A | R) = \frac{\mathbb{P}(A \cap R)}{\mathbb{P}(R)} = \frac{\frac{7}{10} \cdot \frac{1}{2}}{\frac{7}{10} \cdot \frac{1}{2} + \frac{1}{10} \cdot \frac{1}{2}} = \frac{\frac{7}{20}}{\frac{8}{20}} = \frac{7}{8} $

    \newpage 

    \Aufgabe{4}
    \begin{enumerate}[(a)]
        \item

        $ \displaystyle \mathbb{P}(F | \text{Mo}) = \mathbb{P}(F | \text{Di}) = \frac{3}{100} $

        $ \displaystyle \mathbb{P}(F^{\mathsf{c}} | \text{Mo}) = \mathbb{P}(F^{\mathsf{c}} | \text{Di}) = \frac{97}{100} $

        $ \displaystyle \mathbb{P}(F | \text{Mi}) = \mathbb{P}(F | \text{Do}) = \frac{2}{100} $

        $ \displaystyle \mathbb{P}(F^{\mathsf{c}} | \text{Mi}) = \mathbb{P}(F^{\mathsf{c}} | \text{Do}) = \frac{98}{100} $

        $ \displaystyle \mathbb{P}(F | \text{Fr}) \frac{5}{100} $

        $ \displaystyle \mathbb{P}(F^{\mathsf{c}} | \text{Fr}) = \frac{95}{100} $

        \item

        $ \displaystyle\mathbb{P}(F) = \mathbb{P}(\text{Mo} \cap F) + \mathbb{P}(\text{Di} \cap F) + \mathbb{P}(\text{Mi} \cap F) + \mathbb{P}(\text{Do} \cap F) + \mathbb{P}(\text{Fr} \cap F)  $
        
        $ \displaystyle = \frac{1}{7} \cdot \frac{3}{100} + \frac{1}{7} \cdot \frac{3}{100} + \frac{1}{7} \cdot \frac{2}{100} + \frac{1}{7} \cdot \frac{2}{100} + \frac{1}{7} \cdot \frac{5}{100} $

        $ \displaystyle = \frac{3}{140} $

        \item

        $ \displaystyle \mathbb{P}(\text{Fr} | F) = \frac{\mathbb{P}(\text{Fr} \cap F)}{\mathbb{P}(F)} = \frac{\frac{1}{7} \cdot \frac{3}{100}}{\frac{3}{140}} = \frac{420}{2100} = \frac{1}{5} $
    \end{enumerate}
\end{document}