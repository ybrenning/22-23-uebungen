\documentclass[a4paper,12pt]{article}
\usepackage{fancyhdr}
\usepackage{fancyheadings}
\usepackage[ngerman,german]{babel}
\usepackage{german}
\usepackage[utf8]{inputenc}
\usepackage[active]{srcltx}
\usepackage{algorithm}
\usepackage[noend]{algorithmic}
\usepackage{amsmath}
\usepackage{amssymb}
\usepackage{amsthm}
\usepackage{bbm}
\usepackage{enumerate}
\usepackage{graphicx}
\usepackage{ifthen}
\usepackage{listings}
\usepackage{struktex}
\usepackage{hyperref}
\usepackage[T1]{fontenc}
\usepackage{amsfonts}

\newcommand{\Fach}{Wahrscheinlichkeitstheorie}
\newcommand{\Name}{Yannick Brenning, Jean Röther}
\newcommand{\Seminargruppe}{F}
\newcommand{\Matrikelnummer}{3732848, 3796826}
\newcommand{\Semester}{WiSe 22/23}
\newcommand{\Uebungsblatt}{6}

\setlength{\parindent}{0em}
\topmargin -1.0cm
\oddsidemargin 0cm
\evensidemargin 0cm
\setlength{\textheight}{9.2in}
\setlength{\textwidth}{6.0in}

\newcommand{\Aufgabe}[1]{
        {
        \vspace*{0.5cm}
        \textbf{HA #1}
        \vspace*{0.2cm}
    }
}

\hypersetup{
    pdftitle = {\Fach{}: Übungsblatt \Uebungsblatt{}},
    pdfauthor = {\Name},
    pdfborder = {0 0 0}
}

\lstset{
    language=java,
    basicstyle=\footnotesize\tt,
    showtabs=false,
    tabsize=2,
    captionpos=b,
    breaklines=true,
    extendedchars=true,
    showstringspaces=false,
    flexiblecolumns=true,
}

\title{Übungsblatt \Uebungsblatt{}}
\author{\Name{}}

\begin{document}
    \thispagestyle{fancy}
    \lhead{\Fach{} \\ \small \Name{} - \Matrikelnummer{}}
    \rhead{\Semester{} \\  Übungsgruppe \Seminargruppe{}}
    \vspace*{0.2cm}
    \begin{center}
        \LARGE \textbf{Übungsblatt \Uebungsblatt{}}
    \end{center}
    \vspace*{0.2cm}

    \Aufgabe{1}
    \begin{enumerate}[(a)]
        \item 
        $ \displaystyle \mathbb{P}(G_1) = \frac{5}{12} $

        \item
        $ \displaystyle \mathbb{P}(O_1) = \frac{3}{12} = \frac{1}{4} $

        \item
        $ \displaystyle \mathbb{P}(G_2 | R_1) = \frac{\mathbb{P}(G_2 \cap R_1)}{\mathbb{P}(R_1)} = \frac{\frac{5}{11} \cdot \frac{1}{12}}{\frac{1}{12}}  = \frac{5}{132} \cdot \frac{12}{1} = \frac{60}{132} = \frac{5}{11} $

        \item
        $ \displaystyle \mathbb{P}(O_2 | O_1) = \frac{\mathbb{P}(O_2 \cap O_1)}{\mathbb{P}(O_1)} = \frac{\frac{2}{11} \cdot \frac{3}{12}}{\frac{3}{12}} = \frac{6}{132} \cdot \frac{12}{3} = \frac{72}{396} = \frac{2}{11} $

        \item
        $ \displaystyle \mathbb{P}(W_2 \cap O_1) = \frac{3}{11} \cdot \frac{3}{12} = \frac{9}{132} $

        \item
        $ \displaystyle \mathbb{P}(W_2) = \frac{5}{12} \cdot \frac{3}{11} + \frac{3}{12} \cdot \frac{3}{11} + \frac{3}{12} \cdot \frac{2}{11} + \frac{1}{12} \cdot \frac{3}{11} = \frac{33}{132} $

        \item
        $ \displaystyle \mathbb{P}(O_1 | W_2) = \frac{\mathbb{P}(O_1 \cap W_2)}{\mathbb{P}(W_2)} $
    \end{enumerate}
\end{document}