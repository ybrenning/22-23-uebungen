\documentclass[a4paper,12pt]{article}
\usepackage{fancyhdr}
\usepackage{fancyheadings}
\usepackage[ngerman,german]{babel}
\usepackage{german}
\usepackage[utf8]{inputenc}
\usepackage[active]{srcltx}
\usepackage{algorithm}
\usepackage[noend]{algorithmic}
\usepackage{amsmath}
\usepackage{amssymb}
\usepackage{amsthm}
\usepackage{bbm}
\usepackage{enumerate}
\usepackage{graphicx}
\usepackage{ifthen}
\usepackage{listings}
\usepackage{struktex}
\usepackage{hyperref}
\usepackage[T1]{fontenc}
\usepackage{amsfonts}

\newcommand{\Fach}{Wahrscheinlichkeitstheorie}
\newcommand{\Name}{Yannick Brenning, Jean Röther}
\newcommand{\Seminargruppe}{F}
\newcommand{\Matrikelnummer}{3732848, 3796826}
\newcommand{\Semester}{WiSe 22/23}
\newcommand{\Uebungsblatt}{3}

\setlength{\parindent}{0em}
\topmargin -1.0cm
\oddsidemargin 0cm
\evensidemargin 0cm
\setlength{\textheight}{9.2in}
\setlength{\textwidth}{6.0in}

\newcommand{\Aufgabe}[1]{
        {
        \vspace*{0.5cm}
        \textbf{HA #1}
        \vspace*{0.2cm}
    }
}

\hypersetup{
    pdftitle = {\Fach{}: Übungsblatt \Uebungsblatt{}},
    pdfauthor = {\Name},
    pdfborder = {0 0 0}
}

\lstset{
    language=java,
    basicstyle=\footnotesize\tt,
    showtabs=false,
    tabsize=2,
    captionpos=b,
    breaklines=true,
    extendedchars=true,
    showstringspaces=false,
    flexiblecolumns=true,
}

\title{Übungsblatt \Uebungsblatt{}}
\author{\Name{}}

\begin{document}
    \thispagestyle{fancy}
    \lhead{\Fach{} \\ \small \Name{} - \Matrikelnummer{}}
    \rhead{\Semester{} \\  Übungsgruppe \Seminargruppe{}}
    \vspace*{0.2cm}
    \begin{center}
        \LARGE \textbf{Übungsblatt \Uebungsblatt{}}
    \end{center}
    \vspace*{0.2cm}

    \Aufgabe{1} 

    Modellierung als Laplace-Raum mit durchnummiererten Karten:

    $ \Omega = \{\omega = (\omega_1, \dots, \omega_{12}), \omega_i \in \{1, \dots, 48\} \mid \omega_i < \omega_{i + 1} \land i \neq j \Rightarrow \omega_i \neq \omega_j\} $

    $ \displaystyle \Rightarrow \mathbb{P}(\omega) = \frac{1}{|\Omega|} $

    Es handelt sich um eine Kombination ohne Wiederholung, d.h. Ziehen ohne Zur"ucklegen ohne Beachtung der Reihenfolge.

    \bigskip

    Es gibt $ n = 24 \cdot 2 $ Karten, da jede der 24 Karten doppelt vorkommt. Insgesamt wird immer $ k = 12 $ mal gezogen.

    \bigskip

    Daher ergibt sich f"ur das Zufallsexperiment $ |\Omega| = \binom{48}{12} $.
    \begin{enumerate}[(a)]
        \item
        $ A = \{\text{Es werden zwei Kreuzdamen gezogen}\} $

        % $ = \{\omega = \{\omega_1, \dots, \omega_n\} \mid \omega_1 \neq \omega_2, \omega_1, \omega_2 \in \{1, 2\} \land \omega_i \in \{3, \dots, 48\}, i = 3, \dots, 12\} $

        $ \displaystyle \mathbb{P}(A) = \frac{|A|}{|\Omega|} = \frac{\binom{46}{10}}{\binom{48}{12}} = 0.05851063829 \approx 0.059 $

        \item
        $ B = \{\text{Es werden genau f"unf Tr"umpfe gezogen}\} $ 

        Da wir genau f"unf Tr"umpfe ziehen wollen, darf keine von den anderen gezogenen Karten ein Trumpf sein.
        Es werden also $ k = 5 $ von den $ n = 24 $ Tr"umpfen gezogen und die restlichen $ k = 7 $ Karten geh"oren zu den $ n = 24 $ nicht-Tr"umpfen.

        $ \displaystyle \mathbb{P}(B) = \frac{|B|}{|\Omega|} = \frac{\binom{24}{5} \cdot \binom{24}{7}}{\binom{48}{10}} = 0.21115421084 \approx 0.21 $
        
        \item
        $ C = \{\text{Es werden genau drei Pik und 4 Herz gezogen}\} $

        Dieses mal sollen $ k = 3 $ von den $ n = 12 $ verschiedenen Pik-Karten gezogen werden. 
        Das gleiche gilt f"ur die vier gezogenen Herz-Karten, und die restlichen vier sollen schlie\ss lich zu den 24 "ubrigen Symbolen geh"oren.

        $ \displaystyle \mathbb{P}(C) = \frac{|C|}{|\Omega|} = \frac{\binom{12}{3} \cdot \binom{12}{4} \cdot \binom{24}{4}}{\binom{48}{12}} = 0.0664386819 \approx 0.066 $ 

        \item
        $ D = \{\text{Es werden mindestens 10 Tr"umpfe gezogen}\} $

        $ \Leftrightarrow \{\text{Es werden 10, 11, oder 12 Tr"umpfe gezogen}\} $

        $ \Rightarrow \displaystyle \mathbb{P}(D) = \sum_{i = 10}^{12}\frac{\binom{24}{i} \cdot \binom{24}{12-i}}{\binom{48}{12}} $ 

        $ = 0.00776974368 + 0.00085989258 + 0.00003881459 $

        $ = 0.00866845085 \approx 0.0087 $

        \item
        $ E = \{\text{Es werden genau 3 Pik oder genau 4 Herz gezogen}\} $

        $ \displaystyle \mathbb{P}(E) = \frac{\binom{12}{3} \cdot \binom{36}{9}}{\binom{48}{12}} + \frac{\binom{12}{4} \cdot \binom{36}{8}}{\binom{48}{12}} - \frac{\binom{12}{3} \cdot \binom{12}{4} \cdot \binom{24}{5}}{\binom{48}{12}} $

        $ = 0.297286598 + 0.21500191462 - 0.0664386819 $

        $ = 0.44584983072 \approx 0.45 $

        \item
        $ F = \{\text{Es werden genau neun Damen gezogen}\} $

        $ \displaystyle \mathbb{P}(F) = \frac{\binom{8}{9} \cdot \binom{39}{3}}{\binom{48}{12}} = 0 $

        \item
        $ D = \{\text{Es wird ein Kreuz, Pik, Herz oder Karo gezogen}\} $

        Da $ |D| = |\Omega| $, ist die Wahrscheinlichkeit $ \mathbb{P}(D) $ dieses Ereignisses 1.

        Anders gesagt: es gibt keine Karten im Deck, die nicht eine der vier genannten Symbole haben. 
        Somit trifft das Ereignis auf alle Elementarereignisse in $ \Omega $ zu.
    \end{enumerate}

    \Aufgabe{2}
    
    Es sei $ \omega_i $ die ``Nummer'' des Mantels, und $ i $ die Person, die den Mantel erh"alt.

    $ \Omega = \{(\omega_1, \omega_2, \omega_3, \omega_4), \omega_i \in \{1, 2, 3, 4\} \mid i \neq j \Rightarrow \omega_i \neq \omega_j\} $

    \bigskip

    Es handelt sich um eine Variation ohne Wiederholung, d.h. Ziehen ohne Zur"ucklegen mit Beachtung der Reihenfolge.

    \begin{enumerate}[(a)]
        \item 
        $ A = \{\text{Keine der vier Personen erh"alt den eigenen Mantel}\} $

        $ = \{(\omega_1, \omega_2, \omega_3, \omega_4), \omega_i \in \{1, 2, 3, 4\} \mid \omega_i \neq i\} $

        Es werden $ k = 4 $ von den "ubrigen $ n = 4 $ Manteln gezogen.

        $ \displaystyle \Rightarrow |\Omega| = \frac{4!}{0!} = 24 $

        Wir k"onnen $ |A| $ mithilfe des Gegenereignisses bestimmen.

        d.h. $ A^c = \{\text{Mindestens zwei Personen haben den richtigen Mantel}\} $

        $ \displaystyle \mathbb{P}(A) = \frac{|A|}{|\Omega|} = \frac{9}{24} = \frac{3}{8} $

        \item
        $ \Omega = \{\omega = (\omega_1, \omega_2) \mid w_i \in \{1, 2, 3, 4\}, i \neq j \Rightarrow \omega_i \neq \omega_j\} $

        $ \displaystyle P(B) = \frac{1}{\frac{4!}{(4 - 2)!}} = \frac{1}{12} $
    \end{enumerate}

    \Aufgabe{3}
    \begin{enumerate}[(a)]
        \item 
        $ \Omega = \{(\omega_1, \omega_2, \omega_3, \omega_4) \mid \omega_i \in \{1, 2, 3, 4, 5, 6\}\} $

        \item 
        Laplace-Raum:

        $ \Omega = \{(\omega_1, \omega_2) \mid \omega_i \in \{1, 2, 3, 4, 5, 6\}\} $

        Kein Laplace-Raum:

        $ \Omega' = \{\omega = \omega_1 + \omega_2 \mid \omega_i \in \{1, 2, 3, 4, 5, 6\}\} $

        $ \displaystyle \mathbb{P}(\omega) = (\frac{1}{6} \cdot \frac{1}{6})^24  $

        $ \Rightarrow A = \Omega^{24} $

        \item
        Spiel 1:

        $ A = \{\text{Unter den vier W"urfen kommt mindestens eine sechs}\} $

        $ A^c = \{\text{Unter den vier W"urfen kommt keine sechs}\} $

        $ \displaystyle \Rightarrow \mathbb{P}(A) = 1 - \mathbb{P}(A^c) = 1 - (\frac{5}{6})^4 \approx 0.518 $

        $ B = \{\text{24 Doppelw"urfe, darunter mindestens eine Doppelsechs}\} $

        $ B^c = \{\text{24 Doppelw"urfe, darunter keine Doppelsechs}\} $

        $ \displaystyle \mathbb{P}(B) = 1 - \mathbb{P}(B^c) = 1 - (1 - (\frac{1}{6} \cdot \frac{1}{6})^2)^{24} \approx 0.491 $

        \item
        Die Ereigniswahrscheinlichkeiten summieren sich nicht auf, da jedes mal der W"urfel neu geworfen wird.
        Sonst h"atte das Ereignis, bei sechs W"urfen eine sechs zu werfen, die Wahrscheinlichkeit $ \frac{1}{6} \cdot 6  = 1 $.

        Statt zu multiplizieren, m"ussen die Elementarwahrscheinlichkeiten potenziert werden.
    \end{enumerate}
\end{document}