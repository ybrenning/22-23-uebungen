\documentclass[a4paper,12pt]{article}
\usepackage{fancyhdr}
\usepackage{fancyheadings}
\usepackage[ngerman,german]{babel}
\usepackage{german}
\usepackage[utf8]{inputenc}
\usepackage[active]{srcltx}
\usepackage{algorithm}
\usepackage[noend]{algorithmic}
\usepackage{amsmath}
\usepackage{amssymb}
\usepackage{amsthm}
\usepackage{bbm}
\usepackage{enumerate}
\usepackage{graphicx}
\usepackage{ifthen}
\usepackage{listings}
\usepackage{struktex}
\usepackage{hyperref}
\usepackage[T1]{fontenc}
\usepackage{amsfonts}

\newcommand{\Fach}{Wahrscheinlichkeitstheorie}
\newcommand{\Name}{Yannick Brenning, Jean Röther}
\newcommand{\Seminargruppe}{F}
\newcommand{\Matrikelnummer}{3732848, 3796826}
\newcommand{\Semester}{WiSe 22/23}
\newcommand{\Uebungsblatt}{2}

\setlength{\parindent}{0em}
\topmargin -1.0cm
\oddsidemargin 0cm
\evensidemargin 0cm
\setlength{\textheight}{9.2in}
\setlength{\textwidth}{6.0in}

\newcommand{\Aufgabe}[1]{
        {
        \vspace*{0.5cm}
        \textbf{HA #1}
        \vspace*{0.2cm}
    }
}

\hypersetup{
    pdftitle = {\Fach{}: Übungsblatt \Uebungsblatt{}},
    pdfauthor = {\Name},
    pdfborder = {0 0 0}
}

\lstset{
    language=java,
    basicstyle=\footnotesize\tt,
    showtabs=false,
    tabsize=2,
    captionpos=b,
    breaklines=true,
    extendedchars=true,
    showstringspaces=false,
    flexiblecolumns=true,
}

\title{Übungsblatt \Uebungsblatt{}}
\author{\Name{}}

\begin{document}
    \thispagestyle{fancy}
    \lhead{\Fach{} \\ \small \Name{} - \Matrikelnummer{}}
    \rhead{\Semester{} \\  Übungsgruppe \Seminargruppe{}}
    \vspace*{0.2cm}
    \begin{center}
        \LARGE \textbf{Übungsblatt \Uebungsblatt{}}
    \end{center}
    \vspace*{0.2cm}

    \Aufgabe{1}
    \begin{enumerate}[(a)]
        \item
        \begin{enumerate}[1.]
            \item
            Emil w"ahlt zuf"allig f"unf Speisen, welche er in der bestellten bzw. ausgew"ahlten Reihenfolge
            essen m"ochte und somit auch in der gegebenen Reihenfolge vom Koch auf das F"orderband gestellt werden sollen.
            Emil ist es egal, wenn eine Speise mehrfach vorkommt. ich weiss gerade
            \item
            Emil w"ahlt zuf"allig f"unf voneinander verschiedene Speisen, welche in der gegebenen Reihenfolge
            auf das Band gestellt und gegessen werden sollen.
            \item
            Emil m"ochte f"unf voneinander verschiedene Speisen zuf"allig bestellen.
            Emil hat kein Problem damit, bspw. den Nachtisch vor der Hauptspeise zu essen,
            also k"onnen die Gerichte vom Koch beliebig auf das Band gestellt werden.
            \item
            Weil Emil sehr gro\ss en Hunger hat, will er in beliebiger Reihenfolge zuf"allige Gerichte essen, die sich
            ggf. auch doppeln k"onnen.
        \end{enumerate}

        \item
        \begin{enumerate}[1.]
            \item
            \begin{flalign*}
                n^k & = 8^5 && \\
            & = 32768
            \end{flalign*}

            \item
            \begin{flalign*}
                \frac{n!}{(n - k)!} & = \frac{8!}{(8 - 5)!} && \\
                & = 6720
            \end{flalign*}

            \item
            \begin{flalign*}
                \binom{n}{k} & = \frac{n!}{(n - k)! k!} && \\
                & = \frac{8!}{(8 - 5)! 5!} && \\
                & = 56
            \end{flalign*}

            \item
            \begin{flalign*}
                \binom{n + k - 1}{k} & = \binom{8 + 5 - 1}{5} && \\
                & = \frac{(8 + 5 - 1)!}{(8 + 5 - 1 - 5)! 5!} && \\
                & = 792
            \end{flalign*}
        \end{enumerate}
    \end{enumerate}

    \Aufgabe{2}
    \begin{enumerate}[(a)]
        \item
        \begin{enumerate}[i.]
            \item
            Sie m"usste $ 30^2 = 900 $ verschiedene Kombinationen ausprobieren.

            \item
            Es verbleiben $ \frac{30!}{(30 - 2)!}  = 870 $ M"oglichkeiten.
        \end{enumerate}
    \end{enumerate}

\end{document}