\documentclass[a4paper,12pt]{article}
\usepackage{fancyhdr}
\usepackage{fancyheadings}
\usepackage[ngerman,german]{babel}
\usepackage{german}
\usepackage[utf8]{inputenc}
\usepackage[active]{srcltx}
\usepackage{algorithm}
\usepackage[noend]{algorithmic}
\usepackage{amsmath}
\usepackage{amssymb}
\usepackage{amsthm}
\usepackage{bbm}
\usepackage{enumerate}
\usepackage{graphicx}
\usepackage{ifthen}
\usepackage{listings}
\usepackage{struktex}
\usepackage{hyperref}
\usepackage[T1]{fontenc}
\usepackage{amsfonts}

\newcommand{\Fach}{Wahrscheinlichkeitstheorie}
\newcommand{\Name}{Yannick Brenning, Jean Röther}
\newcommand{\Seminargruppe}{F}
\newcommand{\Matrikelnummer}{3732848, 3796826}
\newcommand{\Semester}{WiSe 22/23}
\newcommand{\Uebungsblatt}{2}

\setlength{\parindent}{0em}
\topmargin -1.0cm
\oddsidemargin 0cm
\evensidemargin 0cm
\setlength{\textheight}{9.2in}
\setlength{\textwidth}{6.0in}

\newcommand{\Aufgabe}[1]{
        {
        \vspace*{0.5cm}
        \textbf{HA #1}
        \vspace*{0.2cm}
    }
}

\newcommand{\PAufgabe}[1]{
        {
        \vspace*{0.5cm}
        \textbf{PA #1}
        \vspace*{0.2cm}
    }
}

\hypersetup{
    pdftitle = {\Fach{}: Übungsblatt \Uebungsblatt{}},
    pdfauthor = {\Name},
    pdfborder = {0 0 0}
}

\lstset{
    language=java,
    basicstyle=\footnotesize\tt,
    showtabs=false,
    tabsize=2,
    captionpos=b,
    breaklines=true,
    extendedchars=true,
    showstringspaces=false,
    flexiblecolumns=true,
}

\title{Übungsblatt \Uebungsblatt{}}
\author{\Name{}}

\begin{document}
    \thispagestyle{fancy}
    \lhead{\Fach{} \\ \small \Name{} - \Matrikelnummer{}}
    \rhead{\Semester{} \\  Übungsgruppe \Seminargruppe{}}
    \vspace*{0.2cm}
    \begin{center}
        \LARGE \textbf{Übungsblatt \Uebungsblatt{}}
    \end{center}
    \vspace*{0.2cm}

    \PAufgabe{1}
    \begin{enumerate}[(a)]
        \item 

        Ziehen ohne Zur"ucklegen mit Beachtung der Reihenfolge:

        $ \displaystyle \frac{n!}{(n - k)!} = \frac{6!}{(6 - 4)!} = 360 $

        \item

        Ziehen ohne Zur"ucklegen ohne Beachtung der Reihenfolge:

        $ \displaystyle \binom{n}{k} = \frac{6!}{4!(6 - 4)!} = 15 $

        \item

        Kombinationen f"ur einen Spieler 
        
        (Ziehen ohne Zur"ucklegen ohne Beachtung der Reihenfolge):

        $ \displaystyle \binom{n}{k} = \frac{48!}{12!(48 - 12)!} = 69668534468 $

        \bigskip

        Insgesamte Kombinationen 
        
        (Ziehen ohne Zur"ucklegen mit Beachtung der Reihenfolge):
        
        $ \displaystyle \frac{4!}{(4 - 4)!} = 4! = 24 $

        $ \Rightarrow 24 \cdot 69668534468 $ M"oglichkeiten insgesamt.

        \item

        Ziehen mit Zur"ucklegen mit Beachtung der Ziehungsreihenfolge:

        $ \displaystyle n^k = 4^5 = 1024 $
    \end{enumerate}

    \Aufgabe{1}
    \begin{enumerate}[(a)]
        \item
        \begin{enumerate}[1.]
            \item
            Emil w"ahlt zuf"allig f"unf Speisen, welche er in der bestellten bzw. ausgew"ahlten Reihenfolge
            essen m"ochte und somit auch in der gegebenen Reihenfolge vom Koch auf das F"orderband gestellt werden sollen.
            Emil ist es egal, wenn eine Speise mehrfach vorkommt.
            \item
            Emil w"ahlt zuf"allig f"unf voneinander verschiedene Speisen, welche in der gegebenen Reihenfolge
            auf das Band gestellt und gegessen werden sollen.
            \item
            Emil m"ochte f"unf voneinander verschiedene Speisen zuf"allig bestellen.
            Emil hat kein Problem damit, bspw. den Nachtisch vor der Hauptspeise zu essen,
            also k"onnen die Gerichte vom Koch beliebig auf das Band gestellt werden.
            \item
            Weil Emil sehr gro\ss en Hunger hat, will er in beliebiger Reihenfolge zuf"allige Gerichte essen, die sich
            ggf. auch doppeln k"onnen.
        \end{enumerate}

        \item
        \begin{enumerate}[1.]
            \item
            \begin{flalign*}
                n^k & = 8^5 && \\
            & = 32768
            \end{flalign*}

            \item
            \begin{flalign*}
                \frac{n!}{(n - k)!} & = \frac{8!}{(8 - 5)!} && \\
                & = 6720
            \end{flalign*}

            \item
            \begin{flalign*}
                \binom{n}{k} & = \frac{n!}{(n - k)! k!} && \\
                & = \frac{8!}{(8 - 5)! 5!} && \\
                & = 56
            \end{flalign*}

            \item
            \begin{flalign*}
                \binom{n + k - 1}{k} & = \binom{8 + 5 - 1}{5} && \\
                & = \frac{(8 + 5 - 1)!}{(8 + 5 - 1 - 5)! 5!} && \\
                & = 792
            \end{flalign*}
        \end{enumerate}
    \end{enumerate}

    \Aufgabe{2}
    \begin{enumerate}[(a)]
        \item
        \begin{enumerate}[i.]
            \item
            Sie m"usste $ 30^2 = 900 $ verschiedene Kombinationen ausprobieren.

            \item
            Es verbleiben $ \frac{30!}{(30 - 2)!}  = 870 $ M"oglichkeiten.

            \item 
            Es verbleiben nun $ 9 \cdot 6 = 54 $ verschiedene M"oglichkeiten wenn doppelte Zahlen vorkommen k"onnen, andernfalls sind es $ 54 - 6 = 48 $.

            \item
            Wenn die Reihenfolge unklar ist, gibt es $ (9 \cdot 6) \cdot 2 - 6^2 = 72 $ M"oglichkeiten mit und $ 72 - 6 = 66 $ M"oglichkeiten ohne Doppelziffern.
        \end{enumerate}

        \item
        \begin{enumerate}[i.]
            \item 
            Es gibt insgesamt $ \binom{8}{5} = 56 $ verschiedene Kombinationsm"oglichkeiten.

            \item 
            Wenn Lucie eine Gewinnerin ist, dann gibt es noch $ \binom{8 - 1}{5 - 1} = \binom{7}{4} = 35 $ verschiedene M"oglichkeiten.
            
            \item
            Somit ist die Wahrscheinlichkeit, dass Lucie eine der M"unzen zieht genau $ \frac{35}{56} = 0.625 $.
        \end{enumerate}

        \item
        Mit dieser Bedingung w"urde es sich nun um eine Kombination mit Wiederholung (Ziehen mit Zur"ucklegen und ohne Beachtung der Reihenfolge) handeln.
        Somit g"abe es nun $ \binom{8 + 5 - 1}{5} = \binom{12}{5} = 792 $ verschiedene Gewinnverteilungen.
    \end{enumerate}

    \Aufgabe{3}

    Wir wissen, dass ein Zufallsexperiment ohne Zur"ucklegen und ohne Beachtung der Ziehungsreihenfolge $ \binom{n}{k} $ m"ogliche Ziehungsergebnisse
    hat bei einem Stichprobenumfang von $ k $ und einer Grundgesamtheit von $ n $.
    
    \bigskip

    Wenn wir den Fall betrachten, dass eine Urne $ n $ Kugeln mit der Beschriftung $ a $ und ebenso viele Kugeln mit der Beschriftung $ b $ enth"alt, und wir ziehen $ k = n $ mal:

    \bigskip

    Wir k"onnen die Ereignismenge $ \Omega $ zusammengefasst wie folgt ausdr"ucken:

    $ \Omega = \{(a^n), (a^{n - 1}b^1), (a^{n - 2}b^2), \dots, (a^1b^{n - 1}), (b^n)\} $

    \bigskip

    Wenn wir in diesem Urnenmodell nun alle Ereignisse z"ahlen und summieren:
    \begin{flalign*}
        & \binom{n}{0}a^n + \binom{n}{1}a^{n - 1}b^1 + \binom{n}{2}a^{n - 2}b^2 + \dots + \binom{n}{n - 1}a^1b^{n - 1} + \binom{n}{n}b^n && \\
        & \Leftrightarrow \sum\limits_{k = 0}^{n} \binom{n}{k} a^{n - k}b^k && \\
        & = a^n + n \cdot a^{n - 1}b^1 + \dots + n \cdot a^1b^{n - 1} + b^n && \\
        & = (a + b)^n
    \end{flalign*}

    Beispielsweise mit $ n = 3 $ gibt es $ a^3 $ einmal, $ a^2b $ und $ ab^2 $ jeweils dreimal und $ b^3 $ einmal. 
    Summiert ergibt das:

    $ a^3 + 3a^2b + 3ab^2 + b^3 = (a + b)^3  = \binom{3}{0}a^3 + \binom{3}{1}a^2b + \binom{3}{2}ab^2 + \binom{3}{3}b^3 = \sum\limits_{k = 0}{3}\binom{n}{k}a^{3 - k}b^k $

    % \Aufgabe{4} \\
    % Mit Zur"ucklegen: $ 3^5 = 243 $ vs $ \binom{5 + 3 - 1}{3} = \frac{7!}{3!4!} = 35 $
    % Ohne Zur"ucklegen $ \binom{15}{5} = \frac{15!}{5!10!} = 3003 $ vs $ \frac{15!}{(15 - 5)!} = 360360 $

\end{document}