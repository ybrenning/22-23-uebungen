\documentclass[a4paper,12pt]{article}
\usepackage{fancyhdr}
\usepackage{fancyheadings}
\usepackage[ngerman,german]{babel}
\usepackage{german}
\usepackage[utf8]{inputenc}
\usepackage[active]{srcltx}
\usepackage{algorithm}
\usepackage[noend]{algorithmic}
\usepackage{amsmath}
\usepackage{amssymb}
\usepackage{amsthm}
\usepackage{bbm}
\usepackage{enumerate}
\usepackage{graphicx}
\usepackage{ifthen}
\usepackage{listings}
\usepackage{struktex}
\usepackage{hyperref}
\usepackage[T1]{fontenc}
\usepackage{amsfonts}

\newcommand{\Fach}{Wahrscheinlichkeitstheorie}
\newcommand{\Name}{Yannick Brenning, Jean Röther}
\newcommand{\Seminargruppe}{F}
\newcommand{\Matrikelnummer}{3732848, 3796826}
\newcommand{\Semester}{WiSe 22/23}
\newcommand{\Uebungsblatt}{6}

\setlength{\parindent}{0em}
\topmargin -1.0cm
\oddsidemargin 0cm
\evensidemargin 0cm
\setlength{\textheight}{9.2in}
\setlength{\textwidth}{6.0in}

\newcommand{\Aufgabe}[1]{
        {
        \vspace*{0.5cm}
        \textbf{HA #1}
        \vspace*{0.2cm}
    }
}

\hypersetup{
    pdftitle = {\Fach{}: Übungsblatt \Uebungsblatt{}},
    pdfauthor = {\Name},
    pdfborder = {0 0 0}
}

\lstset{
    language=java,
    basicstyle=\footnotesize\tt,
    showtabs=false,
    tabsize=2,
    captionpos=b,
    breaklines=true,
    extendedchars=true,
    showstringspaces=false,
    flexiblecolumns=true,
}

\title{Übungsblatt \Uebungsblatt{}}
\author{\Name{}}

\begin{document}
    \thispagestyle{fancy}
    \lhead{\Fach{} \\ \small \Name{} - \Matrikelnummer{}}
    \rhead{\Semester{} \\  Übungsgruppe \Seminargruppe{}}
    \vspace*{0.2cm}
    \begin{center}
        \LARGE \textbf{Übungsblatt \Uebungsblatt{}}
    \end{center}
    \vspace*{0.2cm}

    \Aufgabe{2}

    \begin{enumerate}
        \item 
        $ \mathbb{E}(X) = \mathbb{P}(X = 1) \cdot 3 + \mathbb{P}(X = 2) \cdot 1 + \mathbb{P}(X = 3) \cdot (-0.60) + \mathbb{P}(X \geq 4) \cdot (-1) $ \\

        Es handelt sich um eine geometrische Verteilung: \\
        $ G(p) = \mathbb{P}(X = n) = p(1 - p)^{n - 1} $
        \begin{flalign*}
            \Rightarrow \mathbb{P}(X = 1) & = \frac{1}{8} \cdot (1 - \frac{1}{8})^{1 - 1} && \\
            & = \frac{1}{8} \cdot (\frac{7}{8})^0 && \\
            & = \frac{1}{8} && \\ \\
            \mathbb{P}(X = 2) & = \frac{1}{8} \cdot (1 - \frac{1}{8})^{2 - 1} && \\
            & = \frac{1}{8} \cdot (\frac{7}{8})^1 && \\
            & = \frac{7}{64} && \\ \\
            \mathbb{P}(X = 3) & = \frac{1}{8} \cdot (1 - \frac{1}{8})^{3 - 1} && \\
            & = \frac{1}{8} \cdot (\frac{7}{8})^2 && \\
            & = \frac{49}{512}
        \end{flalign*}

        \begin{flalign*}
            \Rightarrow \mathbb{P}(X \geq 4) & = 1 - \mathbb{P}(X < 4) && \\
            & = 1 - \sum_{n = 1}^{3} \mathbb{P}(X = n) && \\
            & = 1 - (\frac{1}{8} + \frac{7}{64} + \frac{49}{512}) && \\
            & = \frac{343}{512}
        \end{flalign*}

        \begin{flalign*}
            \Rightarrow \mathbb{E}(X) & = \frac{1}{8} \cdot 3 + \frac{7}{64} \cdot 1 + \frac{49}{512} \cdot (-0.60) + \frac{343}{512} \cdot (-1) && \\
            & = \frac{3}{8} + \frac{7}{64} - \frac{294}{5120} - \frac{343}{512} && \\
            & \approx -0.24
        \end{flalign*}

        Der Einsatz ist nicht fair, da der Erwartungswert f"ur unseren Gewinn negativ ist, und somit im Durchschnitt Verlust gemacht wird. \\
        Das Spiel ist ``fair'', wenn der Erwartungswert (also der durchschnittliche Gewinn) gleich null ist: \\

        $ \mathbb{E}(X) = 0 $ \\
        \begin{flalign*}
            0 & = \frac{1}{8} \cdot (4 - x) + \frac{7}{64} \cdot (2 - x) + \frac{49}{512} \cdot (0.40 - x) + \frac{343}{512} \cdot (0 - x) && \\
            & = 64 \cdot (4 - x) + 56 \cdot (2 - x) + 49 \cdot (0.40 - x) - 343x && \\
            & = 256 - 64x + 112 - 56x + \frac{49 \cdot 4}{10} - 49x + 343x && \\
            & = -512x + 368 + \frac{196}{10} && \\
            & = -512x + \frac{3680 + 196}{10} && \\
            & = -512x + \frac{3876}{10} && \\ \\
            \Rightarrow x & = \frac{3876}{10} \cdot \frac{1}{512} && \\
            x & = \frac{3876}{5120} && \\
            x & \approx 0.757
        \end{flalign*}

    \end{enumerate}

\end{document}