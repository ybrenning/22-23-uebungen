\documentclass[a4paper,12pt]{article}
\usepackage{fancyhdr}
\usepackage{fancyheadings}
\usepackage[ngerman,german]{babel}
\usepackage{german}
\usepackage[utf8]{inputenc}
\usepackage[active]{srcltx}
\usepackage{algorithm}
\usepackage[noend]{algorithmic}
\usepackage{amsmath}
\usepackage{amssymb}
\usepackage{amsthm}
\usepackage{bbm}
\usepackage{enumerate}
\usepackage{graphicx}
\usepackage{ifthen}
\usepackage{listings}
\usepackage{struktex}
\usepackage{hyperref}
\usepackage[T1]{fontenc}
\usepackage{amsfonts}

\newcommand{\Fach}{Wahrscheinlichkeitstheorie}
\newcommand{\Name}{Yannick Brenning, Jean Röther}
\newcommand{\Seminargruppe}{F}
\newcommand{\Matrikelnummer}{3732848, 3796826}
\newcommand{\Semester}{WiSe 22/23}
\newcommand{\Uebungsblatt}{6}

\setlength{\parindent}{0em}
\topmargin -1.0cm
\oddsidemargin 0cm
\evensidemargin 0cm
\setlength{\textheight}{9.2in}
\setlength{\textwidth}{6.0in}

\newcommand{\Aufgabe}[1]{
        {
        \vspace*{0.5cm}
        \textbf{HA #1}
        \vspace*{0.2cm}
    }
}

\hypersetup{
    pdftitle = {\Fach{}: Übungsblatt \Uebungsblatt{}},
    pdfauthor = {\Name},
    pdfborder = {0 0 0}
}

\lstset{
    language=java,
    basicstyle=\footnotesize\tt,
    showtabs=false,
    tabsize=2,
    captionpos=b,
    breaklines=true,
    extendedchars=true,
    showstringspaces=false,
    flexiblecolumns=true,
}

\title{Übungsblatt \Uebungsblatt{}}
\author{\Name{}}

\begin{document}
    \thispagestyle{fancy}
    \lhead{\Fach{} \\ \small \Name{} - \Matrikelnummer{}}
    \rhead{\Semester{} \\  Übungsgruppe \Seminargruppe{}}
    \vspace*{0.2cm}
    \begin{center}
        \LARGE \textbf{Übungsblatt \Uebungsblatt{}}
    \end{center}
    \vspace*{0.2cm}

    \Aufgabe{1}

    Es seien $ X, Y $ diskrete Zufallsvariablen, so dass gilt $ \mathbb{E}(X^2) < \infty $ und $ \mathbb{E}(Y^2) < \infty $, sowie $ a, b \in \mathbb{R} $. \\
    Dann gilt:

    \begin{enumerate}[(a)]
        \item
        $ \text{Var}(aX + b) = a^2 \text{Var}(X) + 0 $

        \smallskip

        Beweis:
        \begin{flalign*}
            \text{Var}(aX + b) & = \mathbb{E}((aX + b - \mu)^2) && \\
            \Leftrightarrow \text{Var}(aX + b) & = \mathbb{E}((aX + b - \mathbb{E}(aX + b))^2)
        \end{flalign*}
        Aus der Linearit"at des Erwartungswerts folgt:
        \begin{flalign*}
            \Leftrightarrow \text{Var}(aX + b) & = \mathbb{E}((aX + b - (a\mathbb{E}(X) + b))^2) && \\
            & = \mathbb{E}((aX + b - a\mathbb{E}(X) - b)^2) && \\
            & = \mathbb{E}((aX - a\mathbb{E}(X))^2) && \\
            & = \mathbb{E}(a^2 (X - \mathbb{E}(X)^2) && \\
            \Leftrightarrow \text{Var}(aX + b) & = a^2\mathbb{E}((X - \mu)^2) && \\
            \Leftrightarrow \text{Var}(aX + b) & = a^2\text{Var}(X)
        \end{flalign*}

        Daraus folgt der Beweis f"ur $ \text{SD}(aX + b) = |a| \cdot \text{SD}(X) $:
        \begin{flalign*}
            \text{SD}(aX + b) & = + \sqrt{\text{Var}(aX + b)} && \\
            \Leftrightarrow \text{SD}(aX + b) & = + \sqrt{a^2\text{Var}(X)} && \\
            & = (+ \sqrt{a^2}) \cdot (+ \sqrt{\text{Var}(X)}) && \\
            \Leftrightarrow \text{SD}(aX + b) & = |a| \cdot \text{SD}(X)
        \end{flalign*}

        \item
        $ \text{Var}(X + Y) = \text{Var}(X) + \text{Var}(Y) + 2(\mathbb{E}(XY) - \mathbb{E}(X)\mathbb{E}(Y)) $ 

        \smallskip

        Beweis: 
        \begin{flalign*}
            \text{Var}(X + Y) & = \mathbb{E}((X + Y - \mu)^2) && \\
            \Leftrightarrow \text{Var}(X + Y) & = \mathbb{E}((X + Y - \mathbb{E}(X + Y))^2) && \\
            & = \mathbb{E}((X + Y)^2 - 2(X + Y) \mathbb{E}(X + Y) + \mathbb{E}(X + Y)^2) && \\
            & = \mathbb{E}((X + Y)^2 - \mathbb{E}(X + Y))^2 && \\
            & = \mathbb{E}(X^2) + 2\mathbb{E}(XY) + \mathbb{E}(Y^2) - \mathbb{E}(X + Y)^2 && \\
            & = \mathbb{E}(X^2) + 2\mathbb{E}(XY) + \mathbb{E}(Y^2) - (\mathbb{E}(X) + \mathbb{E}(Y))^2 && \\
            & = \mathbb{E}(X^2) + 2\mathbb{E}(XY) + \mathbb{E}(Y^2) - (\mathbb{E}(X)^2 + 2\mathbb{E}(X)\mathbb{E}(Y) + \mathbb{E}(Y)^2)) && \\
            & = (\mathbb{E}(X^2) - \mathbb{E}(X)^2) + (\mathbb{E}(Y^2) - \mathbb{E}(Y)^2) + (2\mathbb{E}(XY) - 2\mathbb{E}(X)\mathbb{E}(Y)) && \\
            \Leftrightarrow \text{Var}(X + Y) & = \text{Var}(X) + \text{Var}(Y) + 2(\mathbb{E}(XY) - \mathbb{E}(X)\mathbb{E}(Y))
        \end{flalign*}
    \end{enumerate}

    \Aufgabe{2}

    \begin{enumerate}
        \item 
        Im folgenden sei $ X $ der Gewinn aus einer gespielten Runde, d.h. der Preis abz"uglich des Einsatzes.
        Somit ergibt sich der Wertebereich $ X \in \{3, 1, -0.6, -1\} $.
        \begin{flalign*}
            \mathbb{P}(X = 3) = \frac{1}{8} &&
        \end{flalign*}
        \begin{flalign*}
            \mathbb{P}(X = 1) & = \frac{7}{8} \cdot \frac{1}{7} && \\
            & = \frac{7}{56} = \frac{1}{8}
        \end{flalign*}
        \begin{flalign*}
            \mathbb{P}(X = -0.6) & = \frac{7}{8} \cdot \frac{6}{7} \cdot \frac{1}{6} && \\
            & = \frac{42}{336} = \frac{1}{8}
        \end{flalign*}
        \begin{flalign*}
            \mathbb{P}(X = -1) & = 1 - \sum_{n = 1}^{3} \mathbb{P}(X = n) && \\
            & = 1 - \frac{3}{8} && \\
            & = \frac{5}{8}
        \end{flalign*}

        Nun kann der Erwartungswert anhand der Zufallsvariablen (Gewinn pro Runde) bestimmt werden:
        \begin{flalign*}
            \mathbb{E}(X) & = \mathbb{P}(X = 3) \cdot 3 + \mathbb{P}(X = 1) \cdot 1 + \mathbb{P}(X = -0.6) \cdot (-0.6) + \mathbb{P}(X = -1) \cdot (-1) && \\
            & = \frac{1}{8} \cdot 3 + \frac{1}{8} \cdot 1 - \frac{1 \cdot 6}{8 \cdot 10} - \frac{5}{8} && \\
            & = \frac{1}{2} - \frac{6}{80} - \frac{5}{8} && \\
            & = \frac{40}{80} - \frac{6}{80} - \frac{50}{80} && \\
            & = -\frac{2}{10}
        \end{flalign*}

        Der Einsatz ist nicht fair, da der Erwartungswert f"ur unseren Gewinn negativ ist, und somit im Durchschnitt Verlust gemacht wird.

        \newpage

        Das Spiel ist ``fair'', wenn der Erwartungswert (also der durchschnittliche Gewinn) gleich null ist.

        \bigskip

        Es sei nun $ e $ der Einsatz, der pro Runde verlangt wird, welcher vom Preis abgezogen wird um den Gewinn zu bestimmen.
        Jede Zufallsvariable wird anhand des Preises und des Einsatzes bestimmt, also kann die Gleichung angepasst werden:

        \begin{equation*}
        \begin{split}
            \mathbb{E}(X) & = \mathbb{P}(X = (4 - e)) \cdot (4 - e) + \mathbb{P}(X = (2 - e)) \cdot (2 - e) \\ 
            & + \mathbb{P}(X = (0.4 - e)) \cdot (0.4 - e) + \mathbb{P}(X = 0 - e) \cdot (0 - e) \\ \\
        \end{split}
        \end{equation*}
        Um einen fairen Einsatz zu bestimmen setzen wir nun den Erwartungswert gleich null:
        \begin{flalign*}
            \mathbb{E}(X) & = 0 && \\
            \Rightarrow 0 & = \frac{4 - e}{8} + \frac{2 - e}{8} + \frac{0.4 - e}{8} + \frac{-5e}{8} && \\
            & = \frac{6.4 - 8e}{8} && \\
            \Leftrightarrow 0 & = 6.4 - 8e && \\
            0.8 & = e 
        \end{flalign*}

        Ein fairer Einsatz f"ur das Spiel w"are also 0.80 EUR, da somit der Erwartungswert (also der durchschnittliche Gewinn
        des Spiels) null w"are.
    
    \item 
    \begin{equation*}
    \begin{split}
        \text{Var}(X) & = \mathbb{P}(X = 3) \cdot (3 + 0.2)^2 + \mathbb{P}(X = 1) \cdot (1 + 0.2)^2 \\
        & + \mathbb{P}(X = -0.6) \cdot (-0.6 + 0.2)^2 + \mathbb{P}(X = -1) \cdot (-1 + 0.2)^2 \\
        & = \frac{10.24 + 1.44 - 0.16 - 5 \cdot 0.64}{8} \\
        & = 2.29
    \end{split}
    \end{equation*}

    \item 
    Der neue Wertebereich ist $ Y \in \{1, 6.76, 9\} $. \\
    \begin{flalign*}
        \mathbb{P}(Y = 1) & = \mathbb{P}(X = 3) + \mathbb{P}(X = 1) && \\
        & = \frac{2}{8} && \\
        \mathbb{P}(Y = 6.76) & = \mathbb{P}(X = 6.76) && \\
        & = \frac{1}{8} && \\
        \mathbb{P}(Y = 9) & = \mathbb{P}(X = -1) && \\
        & = \frac{5}{8}
    \end{flalign*}
    \begin{flalign*}
        \mathbb{E}(X) & = \mathbb{P}(Y = 1) + \mathbb{P}(Y = 6.76) \cdot 6.76 + \mathbb{P}(Y = 9) \cdot 9 && \\
        & = \frac{2}{8} + \frac{6.76}{8} + \frac{45}{8} && \\
        & = \frac{53.76}{8} && \\
        & = 6.72
    \end{flalign*}

    \item
    Anwendung der Transformationsregel mit $ g(X) = (X - 2)^2 $:
    \begin{equation*}
    \begin{split}
        \mathbb{E}(g(X)) & = \sum_{i \in I} g(x_i) \mathbb{P}(X = x_i) \\
        \Rightarrow \mathbb{E}(Y) & = (3 - 2)^2 \cdot \mathbb{P}(X = 3) + (1 - 2)^2 \cdot \mathbb{P}(X = 1) \\
        & + (-0.6 - 2)^2 \cdot \mathbb{P}(X = -0.6) + (-1 - 2)^2 \cdot \mathbb{P}(X = -1) \\
        & = \frac{1 + 1 + 6.76 + 9 \cdot 5}{8} \\
        & = \frac{53.76}{8} \\
        & = 6.72
    \end{split}
    \end{equation*}

    \end{enumerate}

    \Aufgabe{3}
    \begin{enumerate}
        \item 
        Um zu zeigen, dass $ \mathbb{P} $ die Wahrscheinlichkeitsverteilung von $ X $ ist, m"ussen zwei Bedingungen erf"ullt sein: \\
        \begin{enumerate}[i.]
            \item 
            $ \forall k \in \mathbb{N}: \mathbb{P}(X = k) \geq 0 $

            \item
            $ \sum\limits_{k \in \mathbb{N}} \mathbb{P}(X = k) = 1 $
        \end{enumerate}
        Zun"achst gilt f"ur $ \mathbb{P}(X = k) = p (1 - p)^{k - 1} $ die Bedingung $ p \geq 0 $, da $ p \in (0, 1) $. \\
        Da $ p \geq 0 $, ist $ \mathbb{P}(X = k) $ f"ur alle $ k \in \mathbb{N} $ ebenfalls nicht negativ.
        
        \bigskip

        Ebenfalls gilt $ \sum\limits_{k \in \mathbb{N}} \mathbb{P}(X = k) = \sum\limits_{k \in \mathbb{N}} p (1 - p)^{k - 1} $, sodass:
        \begin{flalign*}
            \sum\limits_{k \in \mathbb{N}} p (1 - p)^{k - 1} & = p \sum\limits{k \in \mathbb{N} (1 - p)^{k - 1}} && \\
            & = p \cdot \frac{1}{1 - (1 - p)} && \\
            & = 1
        \end{flalign*}

        Da beide Bedingungen erf"ullt sind, ist $ \mathbb{P} $ die Wahrscheinlichkeitsverteilung einer Zufallsvariablen $ X $.

        \item
        \begin{flalign*}
            \mathbb{E}(X) & = \sum_{k = 1}^{\infty} kp(1 - p)^{k - 1} && \\
            & = \sum_{k = 0}^{\infty} (k + 1)p(1 - p)^k && \\
            & = \sum_{k = 0}^{\infty} kp(1 - p)^k + \sum_{k = 1}^{\infty} p(1 - p)^{k - 1} && \\
            & = (1 - p) \cdot \mathbb{E}(X) + 1 && \\
            & = \frac{1}{p}
        \end{flalign*}

        \begin{flalign*}
            \text{Var}(X) & = \mathbb{E}(X^2) - \mathbb{E}(X)^2 && \\
            & = p \sum_{k = 1}^{\infty} k^2 (1 - p)^{k - 1} - \frac{1}{p^2} && \\
            & \dots && \\
            & = p \cdot \frac{2}{p^3} - p \cdot \frac{1}{p^2} - \frac{1}{p^2} && \\
            & = \frac{1}{p^2} - \frac{1}{p} 
        \end{flalign*}
    \end{enumerate}


\end{document}