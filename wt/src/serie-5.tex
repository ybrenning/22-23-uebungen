\documentclass[a4paper,12pt]{article}
\usepackage{fancyhdr}
\usepackage{fancyheadings}
\usepackage[ngerman,german]{babel}
\usepackage{german}
\usepackage[utf8]{inputenc}
\usepackage[active]{srcltx}
\usepackage{algorithm}
\usepackage[noend]{algorithmic}
\usepackage{amsmath}
\usepackage{amssymb}
\usepackage{amsthm}
\usepackage{bbm}
\usepackage{enumerate}
\usepackage{graphicx}
\usepackage{ifthen}
\usepackage{listings}
\usepackage{struktex}
\usepackage{hyperref}
\usepackage[T1]{fontenc}
\usepackage{amsfonts}

\newcommand{\Fach}{Wahrscheinlichkeitstheorie}
\newcommand{\Name}{Yannick Brenning, Jean Röther}
\newcommand{\Seminargruppe}{F}
\newcommand{\Matrikelnummer}{3732848, 3796826}
\newcommand{\Semester}{WiSe 22/23}
\newcommand{\Uebungsblatt}{5}

\setlength{\parindent}{0em}
\topmargin -1.0cm
\oddsidemargin 0cm
\evensidemargin 0cm
\setlength{\textheight}{9.2in}
\setlength{\textwidth}{6.0in}

\newcommand{\Aufgabe}[1]{
        {
        \vspace*{0.5cm}
        \textbf{HA #1}
        \vspace*{0.2cm}
    }
}

\newcommand{\PAufgabe}[1]{
        {
        \vspace*{0.5cm}
        \textbf{PA #1}
        \vspace*{0.2cm}
    }
}

\hypersetup{
    pdftitle = {\Fach{}: Übungsblatt \Uebungsblatt{}},
    pdfauthor = {\Name},
    pdfborder = {0 0 0}
}

\lstset{
    language=java,
    basicstyle=\footnotesize\tt,
    showtabs=false,
    tabsize=2,
    captionpos=b,
    breaklines=true,
    extendedchars=true,
    showstringspaces=false,
    flexiblecolumns=true,
}

\title{Übungsblatt \Uebungsblatt{}}
\author{\Name{}}

\begin{document}
    \thispagestyle{fancy}
    \lhead{\Fach{} \\ \small \Name{} - \Matrikelnummer{}}
    \rhead{\Semester{} \\  Übungsgruppe \Seminargruppe{}}
    \vspace*{0.2cm}
    \begin{center}
        \LARGE \textbf{Übungsblatt \Uebungsblatt{}}
    \end{center}
    \vspace*{0.2cm}

    \PAufgabe{1}
    \begin{enumerate}[(a)]
        \item 

        Die Ereignisse sind kausal abh"angig, da der erste Wurf (welcher bei $ B $ sechs sein soll) die Augenzahl mitbestimmt.

        \item

        $ (\Omega, \mathcal{A}, \mathbb{P}) $ mit
        \begin{itemize}
            \item $ \Omega = \{1, 2, 3, 4, 5, 6\}^2 $
            \item $ \mathcal{A} = \mathcal{P}(\Omega) $
            \item $ \mathbb{P}(\omega) = \frac{1}{|\Omega|} = \frac{1}{36} $
        \end{itemize}

        $ \displaystyle \mathbb{P}(A) = \frac{|A|}{|\Omega|} = \frac{6}{36} = \frac{1}{6} $

        $ \displaystyle \mathbb{P}(B) = \frac{|B|}{|\Omega|} = \frac{6}{36} = \frac{1}{6} $ 

        $ \displaystyle \mathbb{P}(A) \cdot \mathbb{P}(B) = \frac{1}{6} \cdot \frac{1}{6} = \frac{1}{36} $

        $ \displaystyle \mathbb{P}(A \cap B) = \frac{|A \cap B|}{|\Omega|} = \frac{1}{36} $

        $ \Rightarrow \mathbb{P}(A) \cdot \mathbb{P}(B) = \mathbb{P}(A \cap B) $ 
        
        Die Ereignisse $ A $ und $ B $ sind also stochastisch unabh"angig.
    \end{enumerate}

    \PAufgabe{2}
    \begin{itemize}
        \item Ergebnis := Kopf oder Zahl, Erfolg := ``Kopf'', d.h. $ p = \frac{1}{2} $
        \item Ergebnis := Kaputt oder funktionsf"ahig, Erfolg := ``Maschine funktionsf"ahig''
        \item Ergebnis := Gerade oder ungerade Augenzahl, Erfolg := ``Gerade Augenzahl'', d.h. $ p = \frac{3}{6} $
        \item Ergebnis := Auto oder keine Auto, Erfolg := ``Spielzeugauto''
        \item Ergebnis := Wirkt oder wirkt nicht, Erfolg := ``Wirkt''
        \item Ergebnis := Geblitzt oder nicht geblitzt werden, Erfolg := ``Nicht geblitzt''
        \item Ergebnis := Test positiv oder negativ, Erfolg := ``Negativ''
    \end{itemize}

    \PAufgabe{3}
    \begin{enumerate}[(a)]
        \item 

        $ (\Omega, \mathcal{A}, \mathbb{P}) $ mit
        \begin{itemize}
            \item $ \Omega = \{1, 2, 3, 4, 5, 6\}^{10} $
            \item $ \mathcal{A} = \mathcal{P}(\Omega) $
            \item $ \mathbb{P}(\omega) = \frac{1}{6^{10}} $
        \end{itemize}

        \item

        $ N \in \{5, 4, 3, 2, 1, 0, -1, -2, -3, -4, -5\} $

        \item
        
        Es beschreibe $ X $ nun die Anzahl an w"urfen, bis 6 getroffen wurde.

        Es handelt sich um eine geometrische Verteilung, d.h. $ X \sim \text{Geom}(p) $.

        $ \displaystyle \mathbb{P}(X = 1) = \frac{5}{6}^{1 - 1} \cdot \frac{1}{6} = \frac{1}{6} \approx 0.167 = \mathbb{P}(N = 5) $

        $ \displaystyle \mathbb{P}(X = 2) = \frac{5}{6}^{2 - 1} \cdot \frac{1}{6} = \frac{5}{36} \approx 0.139 = \mathbb{P}(N = 4) $

        $ \displaystyle \mathbb{P}(X = 3) = \frac{5}{6}^{3 - 1} \cdot \frac{1}{6} = \frac{25}{216} \approx 0.116 = \mathbb{P}(N = 3) $

        $ \displaystyle \mathbb{P}(X = 4) = \frac{5}{6}^{4 - 1} \cdot \frac{1}{6} = \frac{125}{1296} \approx 0.096 = \mathbb{P}(N = 2) $

        $ \displaystyle \mathbb{P}(X = 5) = \frac{5}{6}^{5 - 1} \cdot \frac{1}{6} = \frac{625}{7776} \approx 0.08 = \mathbb{P}(N = 1) $

        $ \displaystyle \mathbb{P}(X = 6) = \frac{5}{6}^{6 - 1} \cdot \frac{1}{6} = \frac{3125}{46656} \approx 0.067 = \mathbb{P}(N = 0) $

        $ \displaystyle \mathbb{P}(X = 7) = \frac{5}{6}^{7- 1} \cdot \frac{1}{6} = \frac{15625}{279936} \approx 0.056 = \mathbb{P}(N = -1) $

        $ \displaystyle \mathbb{P}(X = 8) = \frac{5}{6}^{8 - 1} \cdot \frac{1}{6} = \frac{78125}{1679616} \approx 0.047 = \mathbb{P}(N = -2) $

        $ \displaystyle \mathbb{P}(X = 9) = \frac{5}{6}^{9 - 1} \cdot \frac{1}{6} = \frac{390625}{10077696} \approx 0.039 = \mathbb{P}(N = -3)$

        $ \displaystyle \mathbb{P}(X = 10) = \frac{5}{6}^{10 - 1} \cdot \frac{1}{6} = \frac{1953125}{60466176} \approx 0.032 = \mathbb{P}(N = -4) $

        $ \Rightarrow \mathbb{P}(N = -5) = 1 - \sum\limits_{k = 1}^{10} \mathbb{P}(X = k) \approx 1 - 0.839 = 0.161 $

        \item

        $ A = \{\text{Der Nettogewinn ist negativ}\} $

        $ \mathbb{P}(A) = \sum\limits_{k = 1}^{4} \mathbb{P}(X = -k) \approx 0.056 + 0.047 + 0.039 + 0.032 =  0.174 $
    \end{enumerate}

    \newpage

    \PAufgabe{4}
    \begin{enumerate}[(a)]
        \item 

        $ (\Omega, \mathcal{A}, \mathbb{P}) $ mit
        \begin{itemize}
            \item $ \Omega = \{(\omega_1, \omega_2) \mid \omega_1, \omega_2 \in \{1, 2, 3, 4\}\} $
            \item $ \mathcal{A} = \mathcal{P}(\Omega) $
            \item $ \mathbb{P}(\omega) = \frac{1}{|\Omega|} = \frac{1}{16} $
        \end{itemize}

        \item

        $ X \in \{1, 2, 3, 4, 6, 8, 9, 12, 16\} $

        $ \mathbb{P}(X = 1) = \frac{1}{16} $

        $ \mathbb{P}(X = 2) = \frac{2}{16} $

        $ \mathbb{P}(X = 3) = \frac{2}{16} $

        $ \mathbb{P}(X = 4) = \frac{3}{16} $

        $ \mathbb{P}(X = 6) = \frac{2}{16} $

        $ \mathbb{P}(X = 8) = \frac{2}{16} $

        $ \mathbb{P}(X = 9) = \frac{1}{16} $

        $ \mathbb{P}(X = 12) = \frac{2}{16} $

        $ \mathbb{P}(X = 16) = \frac{1}{16} $

        \item

        $ A = \{\text{Das Produkt der Augenzahlen liegt zwischen 7 und 13}\} $ 

        $ \mathbb{P}(A) = \mathbb{P}(X = 8) + \mathbb{P}(X = 9) + \mathbb{P}(X = 12) = \frac{2 + 1 + 2}{16} = \frac{5}{16} $

    \end{enumerate}

    \Aufgabe{1}

    \begin{enumerate}[(b)]
        \item 

        \textbf{Berechnung:}

        $ \displaystyle \mathbb{P}(A_2) \cdot \mathbb{P}(A_1^{\mathsf{c}}) = (\frac{6}{10} \cdot \frac{5}{9} + \frac{4}{10} \cdot \frac{6}{9}) \cdot (1 - \frac{6}{10}) = \frac{54}{90} \cdot \frac{4}{10} = \frac{216}{900} = 0.24 $

        $ \displaystyle \mathbb{P}(A_2 \cap A_1^{\mathsf{c}}) = \frac{4}{10} \cdot \frac{3}{9} = \frac{12}{90} \approx 0.133 $

        $ \Rightarrow \mathbb{P}(A_2) \cdot \mathbb{P}(A_1^{\mathsf{c}}) \neq \mathbb{P}(A_2 \cap A_1^{\mathsf{c}}) $ 
        
        $\Rightarrow $ Die Ereignisse $ A_2 $ und $ A_1^{\mathsf{c}} $ sind somit nicht unabh"angig.

        \bigskip

        \textbf{Erkl"arung:}

        Die Wahrscheinlichkeit von $ A_2 $ beschreibt die Wahrscheinlichkeit, dass im zweiten Zug ein fehlerfreies Produktionsteil entnommen wird.
        Da es sich um ein Ziehen ohne Zur"ucklegen handelt, sind nach dem ersten Zug unterschiedlich viele Produktionsteile in der Kiste, d.h. vor dem zweiten
        Zug sind in jedem Fall weniger Produktionsteile insgesamt enthalten als vor dem ersten. Die Wahrscheinlichkeiten des zweiten Zuges "andern sich also je nachdem 
        was im ersten Zug entnommen wird, denn nach dem ersten Zug gibt es entweder ein fehlerhaftes oder ein fehlerfreies Teil weniger als davor. 
        
        Somit h"angt $ A_2 $ davon ab, was im ersten zug entnommen wurde, in diesem Fall w"are es ein fehlerhaftes Produktionsteil ($ A_1^{\mathsf{c}} $).
    \end{enumerate}

    \Aufgabe{2}

    \begin{enumerate}[(a)]
        \item 
        
        Wahrscheinlichekitsraum $ (\Omega, \mathcal{A}, \mathbb{P}) $ mit

        \begin{itemize}
            \item 
            $ \Omega = \{0, 1, 2, 3, 4\} $

            \item
            $ \mathcal{A} = \mathcal{P}(\Omega) = \{\emptyset, \{0\}, \{0, 1\}, \{0, 1, 2\}, \dots\} $ 
    
            \item
            $ \displaystyle \mathbb{P}(\omega) = \binom{4}{X} \cdot (\frac{1}{2})^4 $
        \end{itemize}

        \item

        $ X \in \{0, 1, 2, 3, 4\} $

        \bigskip

        \setlength{\tabcolsep}{10pt} % Default value: 6pt
        \renewcommand{\arraystretch}{2.5} % Default value: 1
        \begin{tabular}{c|c|c|c|c|c}
            $ k $ & 0 & 1 & 2 & 3 & 4 \\
            \hline
            $ \displaystyle \mathbb{P}(X = k) $ & $ \displaystyle \frac{1}{16} $ & $ \displaystyle \frac{1}{4} $ & $ \displaystyle \frac{1}{2} $ & $ \displaystyle \frac{1}{4} $ & $ \displaystyle\frac{1}{16} $ \\
        \end{tabular}
    \end{enumerate}

    \Aufgabe{3}
    
    \begin{enumerate}[(a)]
        \item 

        $ (\Omega, \mathcal{A}, \mathbb{P}) $

        \begin{itemize}
            \item $ \Omega = \{-4, -3, -2, -1, 0, 1, 2\} $
            \item $ \mathcal{A} = \mathcal{P}(\Omega) $
            \item $ \displaystyle \mathbb{P}(A) = \sum\limits_{w \in A} \mathbb{P}(A), \mathbb{P}(\Omega) = 1 $
        \end{itemize}

        \item

        $ G \in \{2, 1.5, 1, 0.5, -5\} $

        \bigskip

        \setlength{\tabcolsep}{10pt} % Default value: 6pt
        \renewcommand{\arraystretch}{2.5} % Default value: 1
        \begin{tabular}{c|c|c|c|c|c}
            $ k $ & 2 & 1.5 & 1 & 0.5 & -5 \\
            \hline
            $ \displaystyle \mathbb{P}(G = k) $ & $ \displaystyle \frac{2}{20} $ & $ \displaystyle \frac{1}{20} $ & $ \displaystyle \frac{3}{20} $ & $ \displaystyle \frac{7}{20} $ & $ \displaystyle\frac{7}{20} $ \\
        \end{tabular}

        \bigskip

        \item

        Die einzige M"oglichkeit, dass der Besitzer Geld verliert ist mit $ G = -5 $, d.h. die Wahrscheinlichkeit auf Verlust
        f"ur den Besitzer ist $ \mathbb{P}(G = -5) = \frac{7}{20} $.

        \item

        $ \displaystyle \mathbb{E}(G) = \frac{2}{20} \cdot 2 + \frac{1}{20} \cdot 1.5 + \frac{3}{20} \cdot 1 + \frac{7}{20} \cdot 0.5 + \frac{7}{20} \cdot (-5) $

        $ \Rightarrow \mathbb{E}(G) = -1.15 $

        Der Gewinn des Ladenbesitzers nimmt im Mittel einen negativen Wert an, da der Erwartungswert von $ G $ negativ ist.

        Der Stand ist also auf lange Sicht nicht profitabel.
    \end{enumerate}

    \Aufgabe{4}
    \begin{enumerate}[(a)]
        \item 

        $ (\Omega, \mathcal{A}, \mathbb{P}) $

        \begin{itemize}
            \item $ \Omega = \{1, 2, 3, 4, 5\}^10 $
            \item $ \mathcal{A} = \mathcal{P}(\Omega) $
            \item $ \displaystyle \mathbb{P}(\omega) = \frac{1}{|\Omega|} = \frac{1}{5^{10}} $
        \end{itemize}

        Die Zufallsvariable $ X $ beschreibt die Anzahl an Punkten bzw. Anzahl an richtig beantworteten Aufgaben.
        
        $ X \in \{0, 1, 2, \dots, 10\} $

        % \setlength{\tabcolsep}{10pt} % Default value: 6pt
        % \renewcommand{\arraystretch}{2.5} % Default value: 1
        % \begin{tabular}{c|c|c|c|c|c}
        %     $ k $ & 0 & 1 & 2 & 3 & 4 & 5 & 6 & 7 & 8 & 9 & 10\\
        %     \hline
        %     $ \displaystyle \mathbb{P}(X = k) $ & $ \displaystyle \frac{2}{20} $ & $ \displaystyle \frac{1}{20} $ & $ \displaystyle \frac{3}{20} $ & $ \displaystyle \frac{7}{20} $ & $ \displaystyle\frac{7}{20} $ \\
        % \end{tabular}
    \end{enumerate}
\end{document}