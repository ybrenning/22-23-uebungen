\documentclass[a4paper,12pt]{article}
\usepackage{fancyhdr}
\usepackage{fancyheadings}
\usepackage[ngerman,german]{babel}
\usepackage{german}
\usepackage[utf8]{inputenc}
\usepackage[active]{srcltx}
\usepackage{algorithm}
\usepackage[noend]{algorithmic}
\usepackage{amsmath}
\usepackage{amssymb}
\usepackage{amsthm}
\usepackage{bbm}
\usepackage{enumerate}
\usepackage{graphicx}
\usepackage{ifthen}
\usepackage{listings}
\usepackage{struktex}
\usepackage{hyperref}
\usepackage[T1]{fontenc}
\usepackage{amsfonts}

\newcommand{\Fach}{Wahrscheinlichkeitstheorie}
\newcommand{\Name}{Yannick Brenning, Jean Röther}
\newcommand{\Seminargruppe}{F}
\newcommand{\Matrikelnummer}{3732848, 3796826}
\newcommand{\Semester}{WiSe 22/23}
\newcommand{\Uebungsblatt}{7}

\setlength{\parindent}{0em}
\topmargin -1.0cm
\oddsidemargin 0cm
\evensidemargin 0cm
\setlength{\textheight}{9.2in}
\setlength{\textwidth}{6.0in}

\newcommand{\PAufgabe}[1]{
        {
        \vspace*{0.5cm}
        \textbf{PA #1}
        \vspace*{0.2cm}
    }
}

\hypersetup{
    pdftitle = {\Fach{}: Übungsblatt \Uebungsblatt{}},
    pdfauthor = {\Name},
    pdfborder = {0 0 0}
}

\lstset{
    language=java,
    basicstyle=\footnotesize\tt,
    showtabs=false,
    tabsize=2,
    captionpos=b,
    breaklines=true,
    extendedchars=true,
    showstringspaces=false,
    flexiblecolumns=true,
}

\title{Übungsblatt \Uebungsblatt{}}
\author{\Name{}}

\begin{document}
    \thispagestyle{fancy}
    \lhead{\Fach{} \\ \small \Name{} - \Matrikelnummer{}}
    \rhead{\Semester{} \\  Übungsgruppe \Seminargruppe{}}
    \vspace*{0.2cm}
    \begin{center}
        \LARGE \textbf{Übungsblatt \Uebungsblatt{}}
    \end{center}
    \vspace*{0.2cm}

    \PAufgabe{1}

    \setlength{\tabcolsep}{10pt} % Default value: 6pt
    \renewcommand{\arraystretch}{2.5} % Default value: 1
    \begin{tabular}{|c|c|c|c|c|}
        \hline
        $ X $ \textbackslash $ Y $ & -2 & 0 & 1 & \\
        \hline
        0 & 0 & 0 & $ \displaystyle \frac{3}{16} $ & $ \displaystyle \frac{3}{16} $ \\
        \hline
        1 & $ \displaystyle \frac{3}{16} $ &  $ \displaystyle \frac{3}{16} $ & 0 & $ \displaystyle \frac{6}{16} $ \\
        \hline
        2 & $ \displaystyle \frac{4}{16} $ & $ \displaystyle \frac{3}{16} $ & 0 & $ \displaystyle \frac{7}{16} $ \\
        \hline
         & $ \displaystyle \frac{7}{16} $ & $ \displaystyle \frac{6}{16} $ & $ \displaystyle \frac{3}{16} $ & 1 \\
        \hline
    \end{tabular}

    \bigskip

    Unabh"angigkeit:

    $ \mathbb{P}(X = 0) \cdot \mathbb{P}(Y = -2) = \displaystyle \frac{3}{16} \cdot \frac{7}{16} $

    \medskip

    $ \mathbb{P}(X = 0, Y = -2) = 0 \neq \mathbb{P}(X = 0) \cdot \mathbb{P}(Y = -2) $

    $ \Rightarrow X $ und $ Y $ sind nicht stochastisch unabh"angig.

    \bigskip

    $ \displaystyle \mathbb{E}(X) = 0 \cdot \frac{3}{16} + 1 \cdot \frac{6}{16} + 2 \cdot \frac{7}{16} = \frac{20}{16} $

    \medskip

    $ \displaystyle \mathbb{E}(Y) = -2 \cdot \frac{7}{16} + 0 \cdot \frac{6}{16} + 1 \cdot \frac{3}{16} = - \frac{11}{16} $

    \bigskip

    $ \text{Var}(X) = \mathbb{E}(X^2) - \mathbb{E}(X)^2 $

    \bigskip

    $ \displaystyle \mathbb{E}(X^2) = 0^2 \cdot \frac{3}{16} + 1^2 \cdot \frac{6}{16} + 2^2 \cdot \frac{7}{16} = \frac{34}{16} $

    \medskip

    $ \displaystyle \mathbb{E}(Y^2) = (-2)^2 \cdot \frac{7}{16} + 0^2 \cdot \frac{6}{16} + 1^2 \cdot \frac{3}{16} = \frac{31}{16} $

    \bigskip

    $ \displaystyle \Rightarrow \text{Var}(X) = \frac{34}{16} - (\frac{5}{4})^2 = \frac{34 - 25}{16} = \frac{9}{16} $

    \medskip

    $ \displaystyle \Rightarrow \text{Var}(Y) = \frac{31}{16} + (\frac{11}{16})^2 = \frac{375}{256} $

    \newpage

    $ \text{Cov}(X, Y) = \mathbb{E}(XY) - \mathbb{E}(X) \cdot \mathbb{E}(Y) $ 

    \medskip

    $ Z = XY $, sodass $ Z \in \{-4, -2, 0, 1, 2\} $  

    $ \displaystyle \mathbb{E}(XY) = \mathbb{E}(Z) = -4 \cdot \frac{4}{16} -2 \cdot \frac{3}{16} = \frac{-16 - 6}{16} = - \frac{22}{16} $

    \medskip

    $ \displaystyle \text{Cov}(X, Y) = -\frac{11}{8} - \frac{5}{4} \cdot -(\frac{11}{16}) = -\frac{11}{8} + \frac{55}{64} = - \frac{33}{64} $

    \bigskip

    $ \displaystyle \Rightarrow \rho_{X, Y} = \frac{\text{Cov}(X, Y)}{\sqrt{\text{Var}(X) \cdot \text{Var}(Y)}} $

    \PAufgabe{4}
    \begin{enumerate}[(a)]
        \item 

        $ f $ ist eine Wahrscheinlichkeitsdichte obwohl manche Funktionswerte gr"o\ss er als 1 sind, da Wkt.-Dichtefunktionen mit dem Integral
        die Wahrscheinlichkeiten berechnen, d.h. die Fl"ache unter $ f $ im Intervall $ [0, 1] $ ist in diesem Fall gleich 1. Au\ss erdem ist die Kurve nichtnegativ, was eine weitere Eigenschaft der Wkt.-Dichtefunktionen ist.

        \item

        \begin{flalign*}
            \mathbb{P}(x < 0.1) & = \int_{0}^{0.1} 3(x - 1)^2 \,dx && \\
            & = \left[(x - 1)^3\right]_{0}^{0.1} && \\
            & = (-0.9)^3 - (-1)^3 && \\
            & = 0.271
        \end{flalign*}
        \begin{flalign*}
            \mathbb{P}(x < 0.5) & = \int_{0}^{0.5} 3(x - 1)^2 \,dx && \\
            & = \left[(x - 1)^3\right]_{0}^{0.5} && \\
            & = (-0.5)^3 - (-1)^3 && \\
            & = 0.875
        \end{flalign*}

        \item

        \begin{flalign*}
            \mathbb{E}(X) & = \int_{0}^{1} x \cdot f(x) \,dx && \\
            & =\int_{0}^{1} 3x^3 - 6x^2 + 3x \,dx && \\
            & = \left[\frac{3}{4}x^4 - 2x^3 + \frac{3}{2}x^2\right]_0^1 && \\
            & = \frac{3}{4} - 2 + \frac{3}{2} - 0 && \\
            & = \frac{1}{4}
        \end{flalign*}
        \begin{flalign*}
            \sigma^2 & = \text{Var}(X) && \\
            & = \mathbb{E}(X^2) - \mathbb{E}(X)^2 && \\ \\
            \mathbb{E}(X^2) & = \int_{0}^{1} x^2 \cdot f(x) \,dx && \\
            & = \int_{0}^{1} 3x^4 - 6x^3 + 3x^2 \,dx && \\
            & = \left[\frac{3}{5}x^5 - \frac{3}{2}x^4 + x^3\right]_0^1 && \\
            & = \frac{3}{5} - \frac{3}{2} + 1 - 0 && \\
            & = \frac{1}{10} && \\ \\
            \Rightarrow \text{Var}(X) & = \frac{1}{10} - \frac{1}{16} && \\
            & = \frac{3}{80} && \\ \\
            \Rightarrow \sigma & = \sqrt{\text{Var}(X)} = \sqrt{\frac{3}{80}} \approx 0.19
        \end{flalign*}
        
    \end{enumerate}
\end{document}