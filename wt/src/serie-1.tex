\documentclass[a4paper,12pt]{article}
\usepackage{fancyhdr}
\usepackage{fancyheadings}
\usepackage[ngerman,german]{babel}
\usepackage{german}
\usepackage[utf8]{inputenc}
\usepackage[active]{srcltx}
\usepackage{algorithm}
\usepackage[noend]{algorithmic}
\usepackage{amsmath}
\usepackage{amssymb}
\usepackage{amsthm}
\usepackage{bbm}
\usepackage{enumerate}
\usepackage{graphicx}
\usepackage{ifthen}
\usepackage{listings}
\usepackage{struktex}
\usepackage{hyperref}
\usepackage[T1]{fontenc}
\usepackage{amsfonts}

%%%%%%%%%%%%%%%%%%%%%%%%%%%%%%%%%%%%%%%%%%%%%%%%%%%%%%
%%%%%%%%%%%%%%%%%%%%%%%%%%%%%%%%%%%%%%%%%%%%%%%%%%%%%%
\newcommand{\Fach}{Wahrscheinlichkeitstheorie}
\newcommand{\Name}{Yannick Brenning, Jean Röther}
\newcommand{\Seminargruppe}{F}
\newcommand{\Matrikelnummer}{3732848, 3796826}
\newcommand{\Semester}{WiSe 22/23}
\newcommand{\Uebungsblatt}{1} %  <-- UPDATE ME
%%%%%%%%%%%%%%%%%%%%%%%%%%%%%%%%%%%%%%%%%%%%%%%%%%%%%%
%%%%%%%%%%%%%%%%%%%%%%%%%%%%%%%%%%%%%%%%%%%%%%%%%%%%%%

\setlength{\parindent}{0em}
\topmargin -1.0cm
\oddsidemargin 0cm
\evensidemargin 0cm
\setlength{\textheight}{9.2in}
\setlength{\textwidth}{6.0in}

\newcommand{\Aufgabe}[1]{
        {
        \vspace*{0.5cm}
        \textbf{HA #1}
        \vspace*{0.2cm}
    }
}

\hypersetup{
    pdftitle = {\Fach{}: Übungsblatt \Uebungsblatt{}},
    pdfauthor = {\Name},
    pdfborder = {0 0 0}
}

\lstset{
    language=java,
    basicstyle=\footnotesize\tt,
    showtabs=false,
    tabsize=2,
    captionpos=b,
    breaklines=true,
    extendedchars=true,
    showstringspaces=false,
    flexiblecolumns=true,
}

\title{Übungsblatt \Uebungsblatt{}}
\author{\Name{}}

\begin{document}
    \thispagestyle{fancy}
    \lhead{\Fach{} \\ \small \Name{} - \Matrikelnummer{}}
    \rhead{\Semester{} \\  Übungsgruppe \Seminargruppe{}}
    \vspace*{0.2cm}
    \begin{center}
        \LARGE \textbf{Übungsblatt \Uebungsblatt{}}
    \end{center}
    \vspace*{0.2cm}

    \Aufgabe{4} \\
    Da alle $x_i$ gleich oft vorkommen (n"amlich je ein mal) gibt es keinen Modalwert $x_{mod}$. \\
    \begin{flalign*}
        \bar{x} & = \frac{6.4 + 8.25 + 8.5 + 2.15 + 1.45 + 5.05 + 11.4 + 11.6 + 6.7 + 9.65 + 6.9 + 6.65}{12} && \\
        & = \frac{84.7}{12} \approx 7.06 \\ \\
        x_{med} & = \frac{1}{2} (x_{\frac{n}{2}} + x_{\frac{n}{2} + 1}) = \frac{1}{2} (x_{6} + x_{7}) && \\
        & = \frac{13.6}{2} = 6.8 \\
    \end{flalign*}
    Spannweite $ R = x_{max} - x_{min} = 11.6 - 1.45 = 10.15 $

    % 1.45 2.15 5.05 6.4 6.65 6.7 6.9 8.25 8.5 9.65 11.4 11.6
    % 6.4 8.25 8.5 2.15 1.45 5.05 11.4 11.6 6.7 9.65 6.9 6.65

    \Aufgabe{5}
    \begin{enumerate}[(a)]
        \item
        $ x_{mod} = 200 $ \\
        \begin{flalign*}
            x_{med} & = \frac{1}{2} (x_{\frac{n}{2}} + x_{\frac{n}{2} + 1}) = \frac{1}{2} (x_{5} + x_{6}) && \\
            & = \frac{1}{2} (200 + 200) && \\
            & = 200
        \end{flalign*}
        \begin{flalign*}
            \bar{x} & = \frac{1}{n} \sum_{i=1}^{n} x_i && \\
            & = \frac{200 + 150+ 1200 + 250 + 300 + 200 + 200 + 500 + 100 + 200}{10} && \\
            & = \frac{3300}{10} = 330
        \end{flalign*}

        \item
            \begin{enumerate}[(b1)]
                \item
                Es sei $ x_i $ der abgehobene Betrag des $ i $-ten Kundens, dann seien \\ \\
                $ k_i = 0.1 + 0.01 \cdot x_i $ \\ \\
                 die Kosten für diesen Betrag.
                \begin{flalign*}
                    \bar{k} & = 0.1 + 0.01 \cdot \bar{x} = 0.1 + 0.01 \cdot 330 && \\
                    & = 3.4
                \end{flalign*}

                \item
                \begin{flalign*}
                    k & = 0.1 + 0.01 \cdot x_{med} = 0.1 + 0.01 \cdot 200 && \\
                    & = 2.1
                \end{flalign*}

                \item
                Da der Modus 200 betr"agt und somit 200 der Betrag mit der gr"o\ss ten H"aufigkeit ist, sollte man eine
                Geb"uhr von $ 0.1 + 0.01 \cdot 200 = 2.1 $ tippen, um die besten Chancen zu haben.
            \end{enumerate}
    \end{enumerate}

\end{document}