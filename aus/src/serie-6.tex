\documentclass[a4paper,12pt]{article}
\usepackage{fancyhdr}
\usepackage{fancyheadings}
\usepackage[ngerman,german]{babel}
\usepackage{german}
\usepackage[utf8]{inputenc}
\usepackage[active]{srcltx}
\usepackage{algorithm}
\usepackage[noend]{algorithmic}
\usepackage{amsmath}
\usepackage{amssymb}
\usepackage{amsthm}
\usepackage{bbm}
\usepackage{enumerate}
\usepackage{graphicx}
\usepackage{ifthen}
\usepackage{listings}
\usepackage{struktex}
\usepackage{hyperref}
\usepackage[T1]{fontenc}
\usepackage{amsfonts}
\usepackage{tikz}
\usetikzlibrary{automata, arrows.meta, positioning, calc}

\newcommand{\Fach}{Automaten und Sprachen}
\newcommand{\Name}{Yannick Brenning, Jean R"other}
\newcommand{\Seminargruppe}{C}
\newcommand{\Matrikelnummer}{3732848, 3796826}
\newcommand{\Semester}{WiSe 22/23}
\newcommand{\Uebungsblatt}{6}

\setlength{\parindent}{0em}
\topmargin -1.0cm
\oddsidemargin 0cm
\evensidemargin 0cm
\setlength{\textheight}{9.2in}
\setlength{\textwidth}{6.0in}

\newcommand{\Aufgabe}[1]{
        {
        \vspace*{0.5cm}
        \textbf{Hausaufgabe #1}
        \vspace*{0.2cm}
    }
}

\hypersetup{
    pdftitle = {\Fach{}: Übungsblatt \Uebungsblatt{}},
    pdfauthor = {\Name},
    pdfborder = {0 0 0}
}

\lstset{
    language=java,
    basicstyle=\footnotesize\tt,
    showtabs=false,
    tabsize=2,
    captionpos=b,
    breaklines=true,
    extendedchars=true,
    showstringspaces=false,
    flexiblecolumns=true,
    keepspaces,
}

\title{Übungsblatt \Uebungsblatt{}}
\author{\Name{}}

\begin{document}
\thispagestyle{fancy}
\lhead{\Fach{} \\ \small \Name{} - \Matrikelnummer{}}
\rhead{\Semester{} \\  Übungsgruppe \Seminargruppe{}}
\vspace*{0.2cm}
\begin{center}
    \LARGE \textbf{Übungsblatt \Uebungsblatt{}}
\end{center}
\vspace*{0.2cm}

\Aufgabe{4}

\begin{tabular}{ |c|c|c|c|c|c| } 
    \hline
    $ a $ & $ a $ & $ a $ & $ b $ & $ b $ & $ b $ \\
    \hline
    $ \{T\} $ & $ \{T\} $ & $ \{T\} $ & $ \{V\} $ & $ \{V\} $ & $ \{V\} $ \\
    \hline
    $ \{\} $ & $ \{\} $ & $ \{S, V\} $ & $ \{T\} $ & $ \{T\} $ & \\
    \hline
    $ \{\} $ & $ \{U, V\} $ & $ \{T\} $ & $ \{S, V\} $ &   &  \\
    \hline
    $ \{S, V\} $ & $ \{T\} $ & $ \{S, V, U\} $ &   &   &  \\
    \hline
    $ \{T\} $ & $ \{S, V, U\} $ &   &   &   &  \\
    \hline
    $ \{S, V\} $ &  &   &   &   &  \\
    \hline
\end{tabular}

\bigskip

Da $ S \in N_{1, 6} $, l"asst sich das gegebene Wort $ w_1 = aaabbb $ unter der Grammatik $ G $ aus $ S $ ableiten.
Somit ist $ w_1 \in L(G) $. 

\end{document}
