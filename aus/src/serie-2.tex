\documentclass[a4paper,12pt]{article}
\usepackage{fancyhdr}
\usepackage{fancyheadings}
\usepackage[ngerman,german]{babel}
\usepackage{german}
\usepackage[utf8]{inputenc}
\usepackage[active]{srcltx}
\usepackage{algorithm}
\usepackage[noend]{algorithmic}
\usepackage{amsmath}
\usepackage{amssymb}
\usepackage{amsthm}
\usepackage{bbm}
\usepackage{enumerate}
\usepackage{graphicx}
\usepackage{ifthen}
\usepackage{listings}
\usepackage{struktex}
\usepackage{hyperref}
\usepackage[T1]{fontenc}
\usepackage{amsfonts}
\usepackage{tikz}
\usetikzlibrary{automata, arrows.meta, positioning, calc}

\newcommand{\Fach}{Automaten und Sprachen}
\newcommand{\Name}{Yannick Brenning, Jean R"other}
\newcommand{\Seminargruppe}{C}
\newcommand{\Matrikelnummer}{3732848, 3796826}
\newcommand{\Semester}{WiSe 22/23}
\newcommand{\Uebungsblatt}{2}

\setlength{\parindent}{0em}
\topmargin -1.0cm
\oddsidemargin 0cm
\evensidemargin 0cm
\setlength{\textheight}{9.2in}
\setlength{\textwidth}{6.0in}

\newcommand{\Aufgabe}[1]{
        {
        \vspace*{0.5cm}
        \textbf{Hausaufgabe #1}
        \vspace*{0.2cm}
    }
}

\hypersetup{
    pdftitle = {\Fach{}: Übungsblatt \Uebungsblatt{}},
    pdfauthor = {\Name},
    pdfborder = {0 0 0}
}

\lstset{
    language=java,
    basicstyle=\footnotesize\tt,
    showtabs=false,
    tabsize=2,
    captionpos=b,
    breaklines=true,
    extendedchars=true,
    showstringspaces=false,
    flexiblecolumns=true,
}

\title{Übungsblatt \Uebungsblatt{}}
\author{\Name{}}

\begin{document}
    \thispagestyle{fancy}
    \lhead{\Fach{} \\ \small \Name{} - \Matrikelnummer{}}
    \rhead{\Semester{} \\  Übungsgruppe \Seminargruppe{}}
    \vspace*{0.2cm}
    \begin{center}
        \LARGE \textbf{Übungsblatt \Uebungsblatt{}}
    \end{center}
    \vspace*{0.2cm}

    \Aufgabe{4} \\
    $ \mathcal{A}_1 $ \\
    \begin{tikzpicture} [node distance = 3cm, on grid]
        \node (q0) [state, initial, initial text = {}] {$q_0$};
        \node (q1) [state, above right of = q0] {$q_1$};
        \node (q2) [state, right of = q0] {$q_2$};
        \node (q3) [state, below right = of q0] {$q_3$};
        \node (q4) [state, accepting, right of = q2] {$q_4$};

        \path [-stealth]
        (q0) edge [above left] node {$a$} (q1)
        (q0) edge [above] node {$b$} (q2)
        (q0) edge [below left] node {$c$} (q3)

        (q1) edge [loop above] node {$a, b, c$} ()
        (q2) edge [loop above] node {$a, b, c$} ()
        (q3) edge [loop below] node {$a, b, c$} ()

        (q1) edge [bend left, above right] node {$b, c$} (q4)
        (q2) edge [above] node {$a, c$} (q4)
        (q3) edge [bend right, below right] node {$a, b$} (q4)
    \end{tikzpicture}

    $ \mathcal{A}_ 2 $

    \begin{tikzpicture} [node distance = 2.5cm, on grid]
        \node (q0) [state, initial, initial text = {}] {$q_0$};
        \node (q1) [state, above right of = q0] {$q_1$};
        \node (q2) [state, below right of = q0] {$q_2$};
        \node (q3) [state, above right = of q1] {$q_3$};
        \node (q4) [state, below right of = q2] {$q_4$};
        \node (q5) [state, accepting, below right of = q1] {$q_5$};

        \path [-stealth]
        (q0) edge [above left] node {$a$} (q1)
        (q0) edge [below left] node {$b$} (q2)

        (q1) edge [loop above] node {$b$} ()
        (q2) edge [loop below] node {$a$} ()
        (q3) edge [loop above] node {$b$} ()
        (q4) edge [loop below] node {$a$} ()

        (q1) edge [bend left, above] node {$a$} (q3)
        (q2) edge [bend left, above] node {$b$} (q4)
        (q3) edge [bend left, below] node {$a$} (q1)
        (q4) edge [bend left, below] node {$b$} (q2)

        (q1) edge [above] node {$a$} (q5)
        (q2) edge [below] node {$b$} (q5)
    \end{tikzpicture}

    \newpage

    \Aufgabe{5}
    \begin{enumerate}[(a)]
        \item
        \begin{tikzpicture} [node distance = 2cm, on grid]
            \node (q0) [state, initial, initial text = {}] {$q_0$};
            \node (q1) [state, right = of q0] {$q_1$};
            \node (q2) [state, right = of q1] {$q_2$};
            \node (q3) [state, accepting, right = of q2] {$q_3$};

            \path [-stealth]
            (q0) edge [loop above] node {$a, b$} ()
            (q0) edge [above] node {$a$} (q1)
            (q1) edge [above] node {$b$} (q2)
            (q2) edge [above] node {$b$} (q3)
        \end{tikzpicture}

        \item Potenzmengenkonstruktion: \mbox{} \\ \\
        \setlength{\tabcolsep}{10pt}
        \renewcommand{\arraystretch}{1.5}
        \begin{tabular}{ c|c|c }
              & a & b \\
            \hline
            $S_0 = \{q_0\}$ & $\{q_0, q_1\}$ & $\{q_0\}$ \\
            $S_1 = \{q_0, q_1\}$ & $\{q_0, q_1\}$ & $\{q_0, q_2\}$ \\
            $S_2 = \{q_0, q_2\}$ & $\{q_0, q_1\}$ & $\{q_0, q_3\}$ \\
            $S_3 = \{q_0, q_3\}$ & $\{q_0, q_1\}$ & $\{q_0\}$ \\
        \end{tabular} \bigskip

        $ \Rightarrow \mathcal{P} = \{Q', \Sigma, S_0, \Delta', F'\} $ \\
        $ Q' = \{S_0, S_1, S_2, S_3\}, F' = \{S_3\} $ \\
        $ \Delta' = \{(S_0, a, S_1), (S_0, b, S_0), (S_1, a, S_1), (S_1, b, S_1),
        (S_2, a, S_1), (S_2, b, S_3), (S_3, a, S_1), (S_3, b, S_0)\} $ \\ \\

        \begin{tikzpicture} [node distance = 2cm, on grid]
            \node (S0) [state, initial, initial text = {}] {$S_0$};
            \node (S1) [state, right = of q0] {$S_1$};
            \node (S2) [state, right = of q1] {$S_2$};
            \node (S3) [state, accepting, right = of q2] {$S_3$};

            \path [-stealth]
            (S0) edge [loop above] node {$b$} ()
            (S0) edge [above] node {$a$} (S1)
            (S1) edge [loop above] node {$a$} ()
            (S1) edge [bend left, above] node {$b$} (S2)
            (S2) edge [bend left, above] node {$a$} (S1)
            (S2) edge [above] node {$b$} (S3)
            (S3) edge [below, bend left=50] node {$a$} (S1)
            (S3) edge [below, bend left=60] node {$b$} (S0)
        \end{tikzpicture}
    \end{enumerate}
    
    \Aufgabe{6}
    \begin{enumerate} [(a)]
        \item
        \begin{tikzpicture} [node distance = 2.5cm, on grid]
            \node (q0) [state, initial, initial text = {}] {$q_0$};
            \node (q1) [state, right = of q0] {$q_1$};
            \node (q2) [state, below left = of q1] {$q_2$};
            \node (q3) [state, below right = of q1] {$q_3$};
            \node (q4) [state, accepting, right = of q1] {$q_4$};

            \path [-stealth]
            (q0) edge [above] node {$\epsilon$} (q1)
            (q1) edge [above left] node {$a$} (q2)
            (q2) edge [above] node {$b$} (q3)
            (q3) edge [right] node {$c$} (q1)
            (q1) edge [bend left, above] node {$\epsilon$} (q4)
            (q4) edge [bend left, above] node {$a$} (q1)
        \end{tikzpicture}

        \item
        \begin{tikzpicture} [node distance = 2cm, on grid]
            \node (q0) [state, initial, initial text = {}, accepting] {$q_0$};
            \node (q1) [state, right = of q0] {$q_1$};
            \node (q2) [state, right = of q1] {$q_2$};

            \path [-stealth]
            (q0) edge [loop above] node {$a, c$} ()
            (q0) edge [above] node {$a$} (q1)
            (q1) edge [above] node {$b$} (q2)
            (q2) edge [bend left=40, below] node {$c$} (q0)
        \end{tikzpicture}
    \end{enumerate}

    \Aufgabe{7} \\
    Wir beweisen: $ L $ ist nicht erkennbar.
    Wir f"uhren einen direkten Beweis: \\
    Sei $ n_0 $ eine beliebige nat"urliche Zahl mit $ n_0 \geq 1 $. \\
    Wir w"ahlen $ w = v \cdot v^R $ mit $ v = a^{n_0} b, v^R = b a^{n_0} $.
    Es gilt $ w \in L $ und $ |w| \geq n_0 $. \\
    Sei $ w = xyz $ eine beliebige Zerlegung mit $ y \neq \epsilon, |xy| \leq n_0 $. \\

    F"ur diese Zerlegung gilt: \\
    Da $ |xy| \leq n_0 $, besteht $ xy $ nur aus $ a $s, und es besteht $ y \neq \epsilon $ auch nur aus mindestens $ a $,
    sodass $ |y| > 0$. \\ \\
    F"ur $ k = 2 $ gilt also $ xy^k z = a^{n_0 + |y|} b \cdot b a^{n_0} $. \\
    Es gilt f"ur die obige Zerlegung nicht mehr $ w = v \cdot v^R $, da $ v $ links mehr $ a $s enth"alt als $ v^R $ rechts.
    Dies widerspricht allerdings der geg. Definition des Spiegelwortes,
    n"amlich dass ein beliebiges Spiegelwort $ w^R $ alle Buchstaben von $ w $ enthalten muss. \\ \\
    Somit ist das Wort nicht in der Sprache und f"ur $ k = 2 $ gilt $ xy^k z \notin L $. \\ \\
    Da $ n_0 $ und die Zerlegung $ xyz $ beliebig sind, folgt durch das Pumping-Lemma im Kontrapositiv: $ L $ ist nicht erkennbar.

\end{document}