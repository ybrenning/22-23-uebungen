\documentclass[a4paper,12pt]{article}
\usepackage{fancyhdr}
\usepackage{fancyheadings}
\usepackage[ngerman,german]{babel}
\usepackage{german}
\usepackage[utf8]{inputenc}
\usepackage[active]{srcltx}
\usepackage{algorithm}
\usepackage[noend]{algorithmic}
\usepackage{amsmath}
\usepackage{amssymb}
\usepackage{amsthm}
\usepackage{bbm}
\usepackage{enumerate}
\usepackage{graphicx}
\usepackage{ifthen}
\usepackage{listings}
\usepackage{struktex}
\usepackage{hyperref}
\usepackage[T1]{fontenc}
\usepackage{amsfonts}
\usepackage{tikz}
\usetikzlibrary{automata, arrows.meta, positioning, calc}

\newcommand{\Fach}{Automaten und Sprachen}
\newcommand{\Name}{Yannick Brenning, Jean R"other}
\newcommand{\Seminargruppe}{C}
\newcommand{\Matrikelnummer}{3732848, 3796826}
\newcommand{\Semester}{WiSe 22/23}
\newcommand{\Uebungsblatt}{2}

\setlength{\parindent}{0em}
\topmargin -1.0cm
\oddsidemargin 0cm
\evensidemargin 0cm
\setlength{\textheight}{9.2in}
\setlength{\textwidth}{6.0in}

\newcommand{\Aufgabe}[1]{
        {
        \vspace*{0.5cm}
        \textbf{Hausaufgabe #1}
        \vspace*{0.2cm}
    }
}

\hypersetup{
    pdftitle = {\Fach{}: Übungsblatt \Uebungsblatt{}},
    pdfauthor = {\Name},
    pdfborder = {0 0 0}
}

\lstset{
    language=java,
    basicstyle=\footnotesize\tt,
    showtabs=false,
    tabsize=2,
    captionpos=b,
    breaklines=true,
    extendedchars=true,
    showstringspaces=false,
    flexiblecolumns=true,
}

\title{Übungsblatt \Uebungsblatt{}}
\author{\Name{}}

\begin{document}
    \thispagestyle{fancy}
    \lhead{\Fach{} \\ \small \Name{} - \Matrikelnummer{}}
    \rhead{\Semester{} \\  Übungsgruppe \Seminargruppe{}}
    \vspace*{0.2cm}
    \begin{center}
        \LARGE \textbf{Übungsblatt \Uebungsblatt{}}
    \end{center}
    \vspace*{0.2cm}

    \Aufgabe{4} \\
    $ \mathcal{A}_1 $ \\
    \begin{tikzpicture} [node distance = 3cm, on grid]
        \node (q0) [state, initial, initial text = {}] {$q_0$};
        \node (q1) [state, above right of = q0] {$q_1$};
        \node (q2) [state, right of = q0] {$q_2$};
        \node (q3) [state, below right = of q0] {$q_3$};
        \node (q4) [state, accepting, right of = q2] {$q_4$};

        \path [-stealth]
        (q0) edge [above left] node {$a$} (q1)
        (q0) edge [above] node {$b$} (q2)
        (q0) edge [below left] node {$c$} (q3)

        (q1) edge [loop above] node {$a, b, c$} ()
        (q2) edge [loop above] node {$a, b, c$} ()
        (q3) edge [loop below] node {$a, b, c$} ()

        (q1) edge [bend left, above right] node {$b, c$} (q4)
        (q2) edge [above] node {$a, c$} (q4)
        (q3) edge [bend right, below right] node {$a, b$} (q4)
    \end{tikzpicture} \\

    $ \mathcal{A}_ 2 $ \\

    \begin{tikzpicture} [node distance = 2.5cm, on grid]
        \node (q0) [state, initial, initial text = {}] {$q_0$};
        \node (q1) [state, above right of = q0] {$q_1$};
        \node (q2) [state, below right of = q0] {$q_2$};
        \node (q3) [state, above right = of q1] {$q_3$};
        \node (q4) [state, below right of = q2] {$q_4$};
        \node (q5) [state, accepting, below right of = q1] {$q_5$};

        \path [-stealth]
        (q0) edge [above left] node {$a$} (q1)
        (q0) edge [below left] node {$b$} (q2)

        (q1) edge [loop above] node {$b$} ()
        (q2) edge [loop below] node {$a$} ()
        (q3) edge [loop above] node {$b$} ()
        (q4) edge [loop below] node {$a$} ()

        (q1) edge [bend left, above] node {$a$} (q3)
        (q2) edge [bend left, above] node {$b$} (q4)
        (q3) edge [bend left, below] node {$a$} (q1)
        (q4) edge [bend left, below] node {$b$} (q2)

        (q1) edge [above] node {$a$} (q5)
        (q2) edge [below] node {$b$} (q5)
    \end{tikzpicture}

    \begin{tikzpicture} [node distance = 2cm, on grid]
        \node (q0) [state, initial, initial text = {}] {$q_0$};
        \node (q1) [state, right = of q0] {$q_1$};
        \node (q2) [state, right = of q1] {$q_2$};
        \node (q3) [state, right = of q2] {$q_3$};

        \path [-stealth]
        (q0) edge [loop above] node {$a, b$} ()
        (q0) edge
    \end{tikzpicture}

\end{document}