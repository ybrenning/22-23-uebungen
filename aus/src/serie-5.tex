\documentclass[a4paper,12pt]{article}
\usepackage{fancyhdr}
\usepackage{fancyheadings}
\usepackage[ngerman,german]{babel}
\usepackage{german}
\usepackage[utf8]{inputenc}
\usepackage[active]{srcltx}
\usepackage{algorithm}
\usepackage[noend]{algorithmic}
\usepackage{amsmath}
\usepackage{amssymb}
\usepackage{amsthm}
\usepackage{bbm}
\usepackage{enumerate}
\usepackage{graphicx}
\usepackage{ifthen}
\usepackage{listings}
\usepackage{struktex}
\usepackage{hyperref}
\usepackage[T1]{fontenc}
\usepackage{amsfonts}
\usepackage{tikz}
\usetikzlibrary{automata, arrows.meta, positioning, calc}

\newcommand{\Fach}{Automaten und Sprachen}
\newcommand{\Name}{Yannick Brenning, Jean R"other}
\newcommand{\Seminargruppe}{C}
\newcommand{\Matrikelnummer}{3732848, 3796826}
\newcommand{\Semester}{WiSe 22/23}
\newcommand{\Uebungsblatt}{5}

\setlength{\parindent}{0em}
\topmargin -1.0cm
\oddsidemargin 0cm
\evensidemargin 0cm
\setlength{\textheight}{9.2in}
\setlength{\textwidth}{6.0in}

\newcommand{\Aufgabe}[1]{
        {
        \vspace*{0.5cm}
        \textbf{Hausaufgabe #1}
        \vspace*{0.2cm}
    }
}

\hypersetup{
    pdftitle = {\Fach{}: Übungsblatt \Uebungsblatt{}},
    pdfauthor = {\Name},
    pdfborder = {0 0 0}
}

\lstset{
    language=java,
    basicstyle=\footnotesize\tt,
    showtabs=false,
    tabsize=2,
    captionpos=b,
    breaklines=true,
    extendedchars=true,
    showstringspaces=false,
    flexiblecolumns=true,
    keepspaces,
}

\title{Übungsblatt \Uebungsblatt{}}
\author{\Name{}}

\begin{document}
\thispagestyle{fancy}
\lhead{\Fach{} \\ \small \Name{} - \Matrikelnummer{}}
\rhead{\Semester{} \\  Übungsgruppe \Seminargruppe{}}
\vspace*{0.2cm}
\begin{center}
    \LARGE \textbf{Übungsblatt \Uebungsblatt{}}
\end{center}
\vspace*{0.2cm}

\Aufgabe{4}
% \begin{tabular}{ |c|c|c|c|c|c|c|c|c|c|c|c|c|c|c|c|c| } 
%     \hline
%     \lstinline|var| & \lstinline|:=| & \lstinline|zahl| & \lstinline|;| &  \lstinline|repeat| & \lstinline|var| & \lstinline|:=| & \lstinline|(| & \lstinline|var| & \lstinline|+| & \lstinline|zahl| & \lstinline|)| & \lstinline|until| & \lstinline|var| & \lstinline|=| & \lstinline|var| & \lstinline|end| \\
%     \hline
%     \lstinline|TERM| &   & \lstinline|TERM| &   &   & \lstinline|TERM| &   &   & \lstinline|TERM| &   &  \lstinline|TERM| &   &   &  \lstinline|TERM| &   & \lstinline|TERM| &   \\  
%     \hline
%        &   &   &   &   &   &   &   &   &   &   &   &   &   &   &   &   \\
%     \hline
%     \lstinline|PROG|  &   &   &   &   &   &   &   &   &   &   &   &   &   &   &   &   \\
%     \hline
%        &   &   &   &   &   &   &   &   &   &   &   &   &   &   &   &   \\
%     \hline
%        &   &   &   &   &   &   & \lstinline|TERM|  &   &   &   &   &   &   &   &   &   \\
%     \hline
%        &   &   &   &   &   &   &   &   &   &   &   &   &   &   &   &   \\
%     \hline
%        &   &   &   &   & \lstinline|PROG|  &   &   &   &   &   &   &   &   &   &   &   \\
%     \hline
%        &   &   &   &   &   &   &   &   &   &   &   &   &   &   &   &   \\
%     \hline
%        &   &   &   &   &   &   &   &   &   &   &   &   &   &   &   &   \\
%     \hline
%        &   &   &   &   &   &   &   &   &   &   &   &   &   &   &   &   \\
%     \hline
%        &   &   &   &   &   &   &   &   &   &   &   &   &   &   &   &   \\
%     \hline
%        &   &   &   &   &   &   &   &   &   &   &   &   &   &   &   &   \\
%     \hline
%        &   &   &   & \lstinline|PROG|  &   &   &   &   &   &   &   &   &   &   &   &   \\
%     \hline
%        &   &   &   &   &   &   &   &   &   &   &   &   &   &   &   &   \\
%     \hline
%        &   &   &   &   &   &   &   &   &   &   &   &   &   &   &   &   \\
%     \hline
%        &   &   &   &   &   &   &   &   &   &   &   &   &   &   &   &   \\
%     \hline
%      \lstinline|PROG|  &   &   &   &   &   &   &   &   &   &   &   &   &   &   &   &   \\
%     \hline
% \end{tabular} \\

$ w_1 \in L(G) $, da die Ableitung: \\
$ \text{\lstinline|PROG|} \rightarrow \text{\lstinline|PROG ; PROG|} \rightarrow \text{\lstinline|var := TERM ; PROG|} $ \\
$ \rightarrow \text{\lstinline|var := zahl ; PROG|} \rightarrow \text{\lstinline|var := zahl ; repeat PROG until TERM = TERM end|} $ \\
$ \rightarrow \text{\lstinline|var := zahl ; repeat var := TERM until TERM = TERM end |} $ \\
$ \rightarrow \text{\lstinline|var := zahl ; repeat var := (TERM + TERM) until TERM = TERM end |}  $ \\
$ \rightarrow \text{\lstinline|var := zahl ; repeat var := (var + TERM) until TERM = TERM end |}  $ \\
$ \rightarrow \text{\lstinline|var := zahl ; repeat var := (var + zahl) until TERM = TERM end |}  $ \\
$ \rightarrow \text{\lstinline|var := zahl ; repeat var := (var + zahl) until var = TERM end |}  $ \\
$ \rightarrow \text{\lstinline|var := zahl ; repeat var := (var + zahl) until var = var end |}  $ \\
f"ur $ w_1 $ existiert. \\ 

$ w_2 \notin L(G) $, da dieses Wort Symbole erh"alt, die Nichtterminale Symbole in  $ G $ sind. 
Somit kann keine Ableitung f"ur $ w_2 $ durch diese Grammatik existieren. \\

$ w_3 \notin L(G) $, da die einzige Produktion in $ G $, welche \lstinline|if| erzeugen kann auch ein \lstinline|else| voraussetzt, was in diesem Wort nicht gegeben ist.
Somit kann keine Ableitung f"ur $ w_3 $ durch diese Grammatik existieren. \\

$ w_4 \notin L(G) $, da das Wort ein Symbol enh"alt ($ - $), welches nicht in der Menge der Terminalsymbole $ \Sigma $ der Grammatik $ G $ enthalten ist.

\Aufgabe{5} \\
$ G_1 = (N, \Sigma, P, S) $ mit $ N = \{S, A, B\}, \Sigma = \{a, b\} $ und \\
$ P = \{ S \rightarrow aSb \mid aB \mid bA \mid C$ \\
$A \rightarrow aA \mid a$ \\
$B \rightarrow bB \mid b$ \\
$C \rightarrow \epsilon \mid CC \} $ \\

$ G_2 = (M, \Sigma, P, S) $ mit $ N = \{S\}, \Sigma = \{a, b\} $ und 
$ P = \{S \rightarrow aSbbb \mid bb\} $ \\

$ G_3 = (M, \Sigma, P, S) $ mit $ N = \{S\}, \Sigma = \{a, b\} $ und \\
$ P = \{ S \rightarrow aSb \mid Ab \mid Ba \mid aA \mid bB $
$ A \rightarrow aA \mid a $
$ B \rightarrow bB \mid b \} $
\newpage

\Aufgabe{6}

\begin{enumerate}[(a)]
    \item 
    $ \epsilon $ - "Uberg"ange entfernen: \\
    $ N_1 = \{S, V\}, N_2 = \{S, V, T\}, N_3 = \{S, V, T\} $ \\

    Resultierende Grammatik $ G_1 = (\{S, T, U ,V\}, \{a, b\}, P_1, S_0) $ mit \\
    $ P_1 = \{ S_0 \rightarrow S \mid \epsilon $ \\
    $ S \rightarrow aSb \mid ab \mid T $ \\
    $ T \rightarrow V $ \\
    $ V \rightarrow bSa \mid ba \} $

    \item 
    Kettenregeln entfernen: \\
    $ K_0 = \{(S_0, S_0), (S, S), (T, T), (V, V)\} $ \\
    $ K_1 = \{(S_0, S_0), (S, S), (T, T), (V, V), (S_0, S), (S, T), (T, V)\} $ \\
    $ K_2 = \{(S_0, S_0), (S, S), (T, T), (V, V), (S_0, S), (S, T), (T, V), (S_0, T), (S, V)\} $ \\
    $ K_3 = \{(S_0, S_0), (S, S), (T, T), (V, V), (S_0, S), (S, T), (T, V), (S_0, T), (S, V), (S_0, V)\} $ \\
    $ K_4 = \{(S_0, S_0), (S, S), (T, T), (V, V), (S_0, S), (S, T), (T, V), (S_0, T), (S, V), (S_0, V)\} $ \\

    Resultierende Grammatik $ G_2  = (\{S, T, U ,V\}, \{a, b\}, P_2, S_0) $ mit \\
    $ P_2 = \{ S_0 \rightarrow \epsilon \mid aSb \mid ab \mid bSa \mid ba $ \\
    $ S \rightarrow aSb \mid ab \mid bSa \mid ba $ \\
    $ T \rightarrow bSa \mid ba $ \\
    $ V \rightarrow bSa \mid ba \}$

    \item 
    Chomsky-Normalform: \\
    Resultierende Grammatik $ G_3 = (\{S, T, U, V, C_1, C_2, X_a, X_b\}, \{a, b\}, P_3, S_0) $ mit \\
    $ P_3 = \{ S_0 \rightarrow \epsilon \mid X_a C_1 \mid X_a X_b \mid X_b C_2 \mid X_b X_a $ \\
    $ S \rightarrow X_a C_1 \mid X_a X_b \mid X_b C_2 \mid X_b X_a $ \\
    $ T \rightarrow X_b C_1 \mid X_b X_a $ \\
    $ V \rightarrow X_b C_1 \mid X_b X_a $ \\
    $ C_1 \rightarrow S X_b $ \\
    $ C_2 \rightarrow S X_a $ \\
    $ X_a \rightarrow a $ \\
    $ X_b \rightarrow b \} $

\end{enumerate}

\Aufgabe{7} 

\begin{enumerate}[(a)]
    \item 
    $ G_1 = (\{S, A, B\}, \{a, b\}, P_1, S) $ mit \\
    $ P_1 = \{S \rightarrow aA \mid bB, A \rightarrow aA \mid bA \mid b, B \rightarrow aB \mid bB \mid b\} $

    \newpage

    \item
    \textbf{Induktionsanfang}: \\
    F"ur jede Ableitung in einer Grammatik $ G_0 $ der L"ange 1 existiert eine 
    "aquivalente Ableitung in $ G_1 $, welche ebenfalls die L"ange 1 hat.

    Man betrachte den Basisfall $ G_0 = (\{S\}, \{a\}, \{S \rightarrow a\}, S) $. \\
    Dann ist die dazu "aquivalente rechtslineare Grammatik $ G_1 = (\{S\}, \{a\}, \{S \rightarrow a\}, S) $. \\
    $ \Rightarrow |S \vdash^*_{G_0} w| = |S \vdash^*_{G_1} w| = 1 $ \\

    \textbf{Induktionsschritt}: \\
    Es sei $ G_0 $ eine linkslineare Grammatik und $ w \in L(G_0) $. \\
    Es existiert eine Ableitung von $ w $ in $ G_0 $: \\
    $ S = B_0 \vdash_{G_0} w_1 B_1 \vdash_{G_0} w_2 B_2 \vdash_{G_0} ... \vdash_{G_0} w $ \\

    Eine "aquivalente rechtslineare Ableitung kann erstellt werden, indem die Reihenfolge der Produktionsregeln
    umgekehrt wird, sodass: \\
    $ S = B_0 \dashv_{G_0} w_1 B_1 \dashv_{G_0} w_2 B_2 \dashv_{G_0} ... \dashv_{G_0} w $ \\

    Somit existiert f"ur jede linkslineare Grammatik $ G_0 $ eine "aquivalente rechtslineare Grammatik $ G_1 $. 
\end{enumerate}

\end{document}
