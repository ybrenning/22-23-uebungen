\documentclass[a4paper,12pt]{article}
\usepackage{fancyhdr}
\usepackage{fancyheadings}
\usepackage[ngerman,german]{babel}
\usepackage{german}
\usepackage[utf8]{inputenc}
\usepackage[active]{srcltx}
\usepackage{algorithm}
\usepackage[noend]{algorithmic}
\usepackage{amsmath}
\usepackage{amssymb}
\usepackage{amsthm}
\usepackage{bbm}
\usepackage{enumerate}
\usepackage{graphicx}
\usepackage{ifthen}
\usepackage{listings}
\usepackage{struktex}
\usepackage{hyperref}
\usepackage[T1]{fontenc}
\usepackage{amsfonts}
\usepackage{tikz}
\usetikzlibrary{automata, arrows.meta, positioning, calc}

\newcommand{\Fach}{Automaten und Sprachen}
\newcommand{\Name}{Yannick Brenning, Jean R"other}
\newcommand{\Seminargruppe}{C}
\newcommand{\Matrikelnummer}{3732848, 3796826}
\newcommand{\Semester}{WiSe 22/23}
\newcommand{\Uebungsblatt}{5}

\setlength{\parindent}{0em}
\topmargin -1.0cm
\oddsidemargin 0cm
\evensidemargin 0cm
\setlength{\textheight}{9.2in}
\setlength{\textwidth}{6.0in}

\newcommand{\Aufgabe}[1]{
        {
        \vspace*{0.5cm}
        \textbf{Hausaufgabe #1}
        \vspace*{0.2cm}
    }
}

\hypersetup{
    pdftitle = {\Fach{}: Übungsblatt \Uebungsblatt{}},
    pdfauthor = {\Name},
    pdfborder = {0 0 0}
}

\lstset{
    language=java,
    basicstyle=\footnotesize\tt,
    showtabs=false,
    tabsize=2,
    captionpos=b,
    breaklines=true,
    extendedchars=true,
    showstringspaces=false,
    flexiblecolumns=true,
    keepspaces,
}

\title{Übungsblatt \Uebungsblatt{}}
\author{\Name{}}

\begin{document}
\thispagestyle{fancy}
\lhead{\Fach{} \\ \small \Name{} - \Matrikelnummer{}}
\rhead{\Semester{} \\  Übungsgruppe \Seminargruppe{}}
\vspace*{0.2cm}
\begin{center}
    \LARGE \textbf{Übungsblatt \Uebungsblatt{}}
\end{center}
\vspace*{0.2cm}

\Aufgabe{4} \\
% \begin{tabular}{ |c|c|c|c|c|c|c|c|c|c|c|c|c|c|c|c|c| } 
%     \hline
%     \lstinline|var| & \lstinline|:=| & \lstinline|zahl| & \lstinline|;| &  \lstinline|repeat| & \lstinline|var| & \lstinline|:=| & \lstinline|(| & \lstinline|var| & \lstinline|+| & \lstinline|zahl| & \lstinline|)| & \lstinline|until| & \lstinline|var| & \lstinline|=| & \lstinline|var| & \lstinline|end| \\
%     \hline
%     \lstinline|TERM| &   & \lstinline|TERM| &   &   & \lstinline|TERM| &   &   & \lstinline|TERM| &   &  \lstinline|TERM| &   &   &  \lstinline|TERM| &   & \lstinline|TERM| &   \\  
%     \hline
%        &   &   &   &   &   &   &   &   &   &   &   &   &   &   &   &   \\
%     \hline
%     \lstinline|PROG|  &   &   &   &   &   &   &   &   &   &   &   &   &   &   &   &   \\
%     \hline
%        &   &   &   &   &   &   &   &   &   &   &   &   &   &   &   &   \\
%     \hline
%        &   &   &   &   &   &   & \lstinline|TERM|  &   &   &   &   &   &   &   &   &   \\
%     \hline
%        &   &   &   &   &   &   &   &   &   &   &   &   &   &   &   &   \\
%     \hline
%        &   &   &   &   & \lstinline|PROG|  &   &   &   &   &   &   &   &   &   &   &   \\
%     \hline
%        &   &   &   &   &   &   &   &   &   &   &   &   &   &   &   &   \\
%     \hline
%        &   &   &   &   &   &   &   &   &   &   &   &   &   &   &   &   \\
%     \hline
%        &   &   &   &   &   &   &   &   &   &   &   &   &   &   &   &   \\
%     \hline
%        &   &   &   &   &   &   &   &   &   &   &   &   &   &   &   &   \\
%     \hline
%        &   &   &   &   &   &   &   &   &   &   &   &   &   &   &   &   \\
%     \hline
%        &   &   &   & \lstinline|PROG|  &   &   &   &   &   &   &   &   &   &   &   &   \\
%     \hline
%        &   &   &   &   &   &   &   &   &   &   &   &   &   &   &   &   \\
%     \hline
%        &   &   &   &   &   &   &   &   &   &   &   &   &   &   &   &   \\
%     \hline
%        &   &   &   &   &   &   &   &   &   &   &   &   &   &   &   &   \\
%     \hline
%      \lstinline|PROG|  &   &   &   &   &   &   &   &   &   &   &   &   &   &   &   &   \\
%     \hline
% \end{tabular} \\

$ w_1 \in L(G) $, da die Ableitung: \\
$ \text{\lstinline|PROG|} \rightarrow \text{\lstinline|PROG ; PROG|} \rightarrow \text{\lstinline|var := TERM ; PROG|} $ \\
$ \rightarrow \text{\lstinline|var := zahl ; PROG|} \rightarrow \text{\lstinline|var := zahl ; repeat PROG until TERM = TERM end|} $ \\
$ \rightarrow \text{\lstinline|var := zahl ; repeat var := TERM until TERM = TERM end |} $ \\
$ \rightarrow \text{\lstinline|var := zahl ; repeat var := (TERM + TERM) until TERM = TERM end |}  $ \\
$ \rightarrow \text{\lstinline|var := zahl ; repeat var := (var + TERM) until TERM = TERM end |}  $ \\
$ \rightarrow \text{\lstinline|var := zahl ; repeat var := (var + zahl) until TERM = TERM end |}  $ \\
$ \rightarrow \text{\lstinline|var := zahl ; repeat var := (var + zahl) until var = TERM end |}  $ \\
$ \rightarrow \text{\lstinline|var := zahl ; repeat var := (var + zahl) until var = var end |}  $ \\
f"ur $ w_1 $ existiert. \\ 

$ w_2 \notin L(G) $, da dieses Wort Symbole erh"alt, die Nichtterminale Symbole in  $ G $ sind. 
Somit kann keine Ableitung f"ur $ w_2 $ durch diese Grammatik existieren. \\

$ w_3 \notin L(G) $, da die einzige Produktion in $ G $, welche \lstinline|if| erzeugen kann auch ein \lstinline|else| voraussetzt, was in diesem Wort nicht gegeben ist.
Somit kann keine Ableitung f"ur $ w_3 $ durch diese Grammatik existieren.

\newpage

$ w_4 \notin L(G) $, da das Wort ein Symbol enh"alt ($ - $), welches nicht in der Menge der Terminalsymbole $ \Sigma $ der Grammatik $ G $ enthalten ist.
\end{document}
