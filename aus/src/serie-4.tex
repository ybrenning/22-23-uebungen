\documentclass[a4paper,12pt]{article}
\usepackage{fancyhdr}
\usepackage{fancyheadings}
\usepackage[ngerman,german]{babel}
\usepackage{german}
\usepackage[utf8]{inputenc}
\usepackage[active]{srcltx}
\usepackage{algorithm}
\usepackage[noend]{algorithmic}
\usepackage{amsmath}
\usepackage{amssymb}
\usepackage{amsthm}
\usepackage{bbm}
\usepackage{enumerate}
\usepackage{graphicx}
\usepackage{ifthen}
\usepackage{listings}
\usepackage{struktex}
\usepackage{hyperref}
\usepackage[T1]{fontenc}
\usepackage{amsfonts}
\usepackage{tikz}
\usetikzlibrary{automata, arrows.meta, positioning, calc}

\newcommand{\Fach}{Automaten und Sprachen}
\newcommand{\Name}{Yannick Brenning, Jean R"other}
\newcommand{\Seminargruppe}{C}
\newcommand{\Matrikelnummer}{3732848, 3796826}
\newcommand{\Semester}{WiSe 22/23}
\newcommand{\Uebungsblatt}{4}

\setlength{\parindent}{0em}
\topmargin -1.0cm
\oddsidemargin 0cm
\evensidemargin 0cm
\setlength{\textheight}{9.2in}
\setlength{\textwidth}{6.0in}

\newcommand{\Aufgabe}[1]{
        {
        \vspace*{0.5cm}
        \textbf{Hausaufgabe #1}
        \vspace*{0.2cm}
    }
}

\hypersetup{
    pdftitle = {\Fach{}: Übungsblatt \Uebungsblatt{}},
    pdfauthor = {\Name},
    pdfborder = {0 0 0}
}

\lstset{
    language=java,
    basicstyle=\footnotesize\tt,
    showtabs=false,
    tabsize=2,
    captionpos=b,
    breaklines=true,
    extendedchars=true,
    showstringspaces=false,
    flexiblecolumns=true,
}

\title{Übungsblatt \Uebungsblatt{}}
\author{\Name{}}

\begin{document}
    \thispagestyle{fancy}
    \lhead{\Fach{} \\ \small \Name{} - \Matrikelnummer{}}
    \rhead{\Semester{} \\  Übungsgruppe \Seminargruppe{}}
    \vspace*{0.2cm}
    \begin{center}
        \LARGE \textbf{Übungsblatt \Uebungsblatt{}}
    \end{center}
    \vspace*{0.2cm}

    \Aufgabe{4} \\
    Wir konstruieren den Produktautomaten $ \mathcal{A} $, sodass $ L(\mathcal{A}) = L(\mathcal{A}_1 \cap L(\mathcal{A}_1)) $. \\ \\
    F"ur den Produktautomaten $ \mathcal{A} $ gilt: \\ \\
    $ \mathcal{A} = (Q_1 \times Q_2, \Sigma, (q_{01}, q_{02}), \Delta, F_1 \times F_2) $ \\
    mit $ \Delta = \{((q_1, q_2), a, (q_1', q_2')) | (q_1, a, q_1') \in \Delta_1 \text{ und } (q_2, a, q_2') \in \Delta_2\} $ \\ \\
    
    $ \Rightarrow \mathcal{A} = (Q, \Sigma, (0, A), \Delta, F) $ mit: \\ \\
    $ Q = \{(0, A), (0, B), (0, C), (1, A), (1, B), (1, C)\} $ \\
    $ F = \{(1, C)\} $ \\ \\
    $ \Delta = \{((0, A), a, (0, B)), ((0, B), a, (0 , C)), ((0, C), a (0, A)), $ \\
    $ ((0, A), a, (1, B)), ((0, B), a, (1, C)), ((0, C), a, (1, A)), () $ \\
    $ ((1, A), b (0, A)), ((1, B), b, (0, B)), ((1, B), b, (0, C)), ((1, C), b, (0, C))\} $ \\ \\

    \begin{tikzpicture}
        \node (0A) [state, initial, initial text = {}] {$0, A$};
        \node (1B) [state, right = of 0A] {$1, B$};
        \node (0B) [state, right = of 1B] {$0, B$};
        \node (1C) [state, accepting, right = of 0B] {$1, C$};
        \node (0C) [state, above = of 1B] {$0, C$};
        \node (1A) [state, left = of 0C] {$1, A$};

        \path [-stealth]
        (0A) edge [above] node {$a$} (1B)
        (0A) edge [below, bend right = 60] node {$a$} (0B)
        (1B) edge [above] node {$b$} (0B)
        (1B) edge [left] node {$b$} (0C)
        (0B) edge [above] node {$a$} (0C)
        (0B) edge [above] node {$a$} (1C)
        (1C) edge [above] node {$b$} (0C)
        (0C) edge [above] node {$a$} (1A)
        (0C) edge [above left] node {$a$} (0A)
        (1A) edge [left] node {$b$} (0A)
    \end{tikzpicture}

    \newpage

    \Aufgabe{5}

    Zun"achst werden die unerreichbaren Zust"ande aus dem Automaten entfernt: \\

    \begin{tikzpicture} [node distance = 2cm, on grid]
        \node (0) [state, initial, initial text = {}] {$0$};
        \node (1) [state, below = of 0] {$1$};
        \node (2) [state, accepting, below = of 1] {$2$};
        \node (3) [state, accepting, right = of 0] {$3$};
        \node (4) [state, below = of 3] {$4$};
        \node (5) [state, accepting, below = of 4] {$5$};
        \node (6) [state, right = of 3] {$6$};
        \node (8) [state, right = of 5] {$8$};

        \path [-stealth]
        (0) edge [above] node {$a$} (3)
        (0) edge [right] node {$b$} (1)
        (1) edge [loop left] node {$b$} ()
        (1) edge [right] node {$a$} (2)
        (2) edge [above left] node {$a$} (4)
        (2) edge [below, bend right = 50] node {$b$} (8)
        (3) edge [left, bend right] node {$a, b$} (4)
        (4) edge [right, bend right] node {$a$} (3)
        (4) edge [above left, bend right] node {$b$} (6)
        (5) edge [above, bend left] node {$a, b$} (8)
        (6) edge [loop right] node {$b$} ()
        (6) edge [above, bend right = 50] node {$a$} (0)
        (8) edge [below] node {$a$} (5)
    \end{tikzpicture}

    \begin{itemize}
        \item $ \sim_0 $ hat die Klassen $ F = \{2, 3, 5\} $ und $ Q \setminus F = \{0, 1, 4, 6, 8\} $
        \item Zwischenschritte f"ur $ \sim_1 $: \\
        $ 2 \sim_1 3 \Leftrightarrow 2 \sim_0 3 \land \forall a \in (a, b): \delta(2, a) \sim_0 \delta(3, a) $ \\
        $ \Rightarrow 4 \sim_0 4 $ und $ 4 \sim_0 8 $ \\ \\

        $ 3 \sim_1 5 \Leftrightarrow 3 \sim_0 5 \land \forall a \in (a, b): \delta(3, a) \sim_0 \delta(5, a) $ \\
        $ \Rightarrow 4 \sim_0 8 $ \\ \\

        $ 4 \sim_1 8 \Leftrightarrow 4 \sim_0 8 \land \forall a \in (a, b): \delta(4, a) \sim_0 \delta(8, a) $ \\
        $ \Leftrightarrow 3 \sim_0 5 $ und $ 6 \sim_0 6 $ \\ \\

        $ 0 \sim_1 1 \Leftrightarrow 0 \sim_0 1 \land \forall a \in (a, b): \delta(0, a) \sim_0 \delta(1, a) $ \\
        $ \Rightarrow 3 \sim_0 2 $ und $ 1 \sim_0 1 $ \\ \\

        $ 1 \sim_1 4 \Leftrightarrow 1 \sim_0 4 \land \forall a \in (a, b): \delta(1, a) \sim_0 \delta(4, a) $ \\
        $ \Rightarrow 2 \sim_0 3 $ und $ 1 \sim_0 6 $ \\ \\

        $ 4 \sim_1 6 \Leftrightarrow 4 \sim_0 6 \land \forall a \in (a, b): \delta(4, a) \sim_0 \delta(6, a) $ \\
        $ \Rightarrow 3 \sim_0 0 $ gilt nicht \\ \\

        $ 4 \sim_1 8 \Leftrightarrow 4 \sim_0 8 \land \forall a \in (a, b): \delta(4, a) \sim_0 \delta(8, a) $ \\
        $ \Rightarrow 3 \sim_0 5 $ und  $ 6 \sim_0 6 $ \\ \\
        $ \Rightarrow \sim_1 $ hat die Klassen $ \{2, 3, 5\}, \{0, 1, 4, 8\}, \{6\} $

        \item Zwischenschritte f"ur $ \sim_2 $: \\
        $ 0 \sim_2 1 \Leftrightarrow 0 \sim_1 1 \land \forall a \in (a, b): \delta(0, a) \sim_1 \delta(1, a) $ \\
        $ \Rightarrow 3 \sim_1 2 $ und  $ 1 \sim_1 1 $ \\ \\

        $ 1 \sim_2 4 \Leftrightarrow 1 \sim_1 4 \land \forall a \in (a, b): \delta(1, a) \sim_1 \delta(4, a) $ \\
        $ \Rightarrow 1 \sim_1 6 $ gilt nicht \\ \\

        $ 4 \sim_2 8 \Leftrightarrow 4 \sim_1 8 \land \forall a \in (a, b): \delta(4, a) \sim_1 \delta(8, a) $ \\
        $ \Rightarrow 3 \sim_1 5 $ und  $ 6 \sim_1 6 $ \\ \\

        $ \Rightarrow \sim_2 $ hat die Klassen $ \{2, 3, 5\}, \{0, 1\}, \{4, 8\}, \{6\} $

        \item $ \sim_3 = \sim_2 \Rightarrow \sim_{\mathcal{A}} $
    \end{itemize}

    Quotientenautomat $ \mathcal{A} $:
    \begin{tikzpicture} [node distance = 2cm, on grid]
        \node (01) [state, initial, initial text = {}] {$0, 1$};
        \node (6) [state, above = of 01] {$6$};
        \node (48) [state, right = of 6] {$4, 8$};
        \node (235) [state, accepting, right = of 01] {$2, 3, 5$};

        \path [-stealth]
        (01) edge [above] node {$a$} (235)
        (0) edge [loop below] node {$b$} ()
        (235) edge [left, bend left] node {$a, b$} (48)
        (48) edge [right, bend left] node {$a$} (235)
        (48) edge [above] node {$b$} (6)
        (6) edge [left] node {$a$} (01)
        (6) edge [loop above] node {$b$} ()
    \end{tikzpicture}

    \Aufgabe{6}
    \begin{enumerate}[(a)]
        \item
        Die Sprache $ L_1 $ hat die folgenden Nerode-"Aquivalenzrelationen: \\
        \begin{itemize}
            \item $ [\epsilon]_{L_1} = \{v \in \Sigma^* \; | \; vw \in L \text{ f"ur kein } w \in \Sigma^*\} = \epsilon $
            \item $ [a]_{L_1} = \{v \in \Sigma^* \; | \; vw \in L \text{ gdw. } w \in \Sigma^*\} = (a + b)^* \cdot (a + b) \cdot (a + b)^* $
        \end{itemize}
        Somit hat die Nerode-Rechtskongruenz von $ L_1 $ einen Index von zwei.
        Nach dem Satz von Myhill und Nerode ist $ L_1 $ somit erkennbar.

    \end{enumerate}

    \Aufgabe{7} \\
    Eine Sprache, die $ \epsilon $ nicht enthält, kann immer durch einen NEA mit einem Endzustand dargestellt werden. \\
    Eine Sprache, die $ \epsilon $ enthält, kann immer durch einen NEA mit zwei Endzuständen dargestellt werden. \\ \\

    Dies kann man mit Hilfe der $ \epsilon $-Elimination beweisen. \\ \\

    Mann kann einen NEA mit mehr als 2 Endzuständen in einen $ \epsilon $-NEA mit nur einem Endzustand wie folgt umwandeln: \\
    - Füge einen neuen Zustand hinzu.
    Dieser wird als einziger zum Endzustand. \\
    - Alle bisherigen Endzustände erhalten  $ \epsilon $-Übergänge in den neuen Zustand. \\ \\

    Nun wendet man die $ \epsilon $-Elimination auf den entstandenen $ \epsilon $-NEA an um ihn wieder in einen NEA ohne $ \epsilon $-Übergänge zurückzuwandeln. \\
    Ausnamhe: Wenn $ \epsilon $ in der Sprache enthalten ist, dann wird auch der Startzustand ebenfalls zum Endzustand. \\

    Fazit: \\
    Wenn $ \epsilon $ nicht in der Sprache enthalten ist, dann hat man nach wie vor nur einen Endzustand.
    Wenn $ \epsilon $ in der Sprache enthalten ist, dann kommt ein zweiter hinzu.

\end{document}