\documentclass[a4paper,12pt]{article}
\usepackage{fancyhdr}
\usepackage{fancyheadings}
\usepackage[ngerman,german]{babel}
\usepackage{german}
\usepackage[utf8]{inputenc}
\usepackage[active]{srcltx}
\usepackage{algorithm}
\usepackage[noend]{algorithmic}
\usepackage{amsmath}
\usepackage{amssymb}
\usepackage{amsthm}
\usepackage{bbm}
\usepackage{enumerate}
\usepackage{graphicx}
\usepackage{ifthen}
\usepackage{listings}
\usepackage{struktex}
\usepackage{hyperref}
\usepackage[T1]{fontenc}
\usepackage{amsfonts}
\usepackage{tikz}
\usetikzlibrary{automata, arrows.meta, positioning, calc}

\newcommand{\Fach}{Automaten und Sprachen}
\newcommand{\Name}{Yannick Brenning, Jean R"other}
\newcommand{\Seminargruppe}{C}
\newcommand{\Matrikelnummer}{3732848, 3796826}
\newcommand{\Semester}{WiSe 22/23}
\newcommand{\Uebungsblatt}{3}

\setlength{\parindent}{0em}
\topmargin -1.0cm
\oddsidemargin 0cm
\evensidemargin 0cm
\setlength{\textheight}{9.2in}
\setlength{\textwidth}{6.0in}

\newcommand{\Aufgabe}[1]{
        {
        \vspace*{0.5cm}
        \textbf{Hausaufgabe #1}
        \vspace*{0.2cm}
    }
}

\hypersetup{
    pdftitle = {\Fach{}: Übungsblatt \Uebungsblatt{}},
    pdfauthor = {\Name},
    pdfborder = {0 0 0}
}

\lstset{
    language=java,
    basicstyle=\footnotesize\tt,
    showtabs=false,
    tabsize=2,
    captionpos=b,
    breaklines=true,
    extendedchars=true,
    showstringspaces=false,
    flexiblecolumns=true,
}

\title{Übungsblatt \Uebungsblatt{}}
\author{\Name{}}

\begin{document}
    \thispagestyle{fancy}
    \lhead{\Fach{} \\ \small \Name{} - \Matrikelnummer{}}
    \rhead{\Semester{} \\  Übungsgruppe \Seminargruppe{}}
    \vspace*{0.2cm}
    \begin{center}
        \LARGE \textbf{Übungsblatt \Uebungsblatt{}}
    \end{center}
    \vspace*{0.2cm}

    \Aufgabe{5} \\
    Wir konstruieren den Produktautomaten $ \mathcal{A} $, sodass $ L(\mathcal{A}) = L(\mathcal{A}_1 \cap L(\mathcal{A}_1)) $. \\ \\
    F"ur den Produktautomaten $ \mathcal{A} $ gilt: \\ \\
    $ \mathcal{A} = (Q_1 \times Q_2, \Sigma, (q_{01}, q_{02}), \Delta, F_1 \times F_2) $ \\
    mit $ \Delta = \{((q_1, q_2), a, (q_1', q_2')) | (q_1, a, q_1') \in \Delta_1 \text{ und } (q_2, a, q_2') \in \Delta_2\} $ \\ \\
    
    $ \Rightarrow \mathcal{A} = (Q, \Sigma, (0, A), \Delta, F) $ mit: \\ \\
    $ Q = \{(0, A), (0, B), (0, C), (1, A), (1, B), (1, C)\} $ \\
    $ F = \{(1, C)\} $ \\ \\
    $ \Delta = \{((0, A), a, (0, B)), ((0, B), a, (0 , C)), ((0, C), a (0, A)), $ \\
    $ ((0, A), a, (1, B)), ((0, B), a, (1, C)), ((0, C), a, (1, A)), () $ \\
    $ ((1, A), b (0, A)), ((1, B), b, (0, B)), ((1, B), b, (0, C)), ((1, C), b, (0, C)) $ \\ \\

\end{document}