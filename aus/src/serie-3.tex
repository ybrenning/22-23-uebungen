\documentclass[a4paper,12pt]{article}
\usepackage{fancyhdr}
\usepackage{fancyheadings}
\usepackage[ngerman,german]{babel}
\usepackage{german}
\usepackage[utf8]{inputenc}
\usepackage[active]{srcltx}
\usepackage{algorithm}
\usepackage[noend]{algorithmic}
\usepackage{amsmath}
\usepackage{amssymb}
\usepackage{amsthm}
\usepackage{bbm}
\usepackage{enumerate}
\usepackage{graphicx}
\usepackage{ifthen}
\usepackage{listings}
\usepackage{struktex}
\usepackage{hyperref}
\usepackage[T1]{fontenc}
\usepackage{amsfonts}
\usepackage{tikz}
\usetikzlibrary{automata, arrows.meta, positioning, calc}

\newcommand{\Fach}{Automaten und Sprachen}
\newcommand{\Name}{Yannick Brenning, Jean R"other}
\newcommand{\Seminargruppe}{C}
\newcommand{\Matrikelnummer}{3732848, 3796826}
\newcommand{\Semester}{WiSe 22/23}
\newcommand{\Uebungsblatt}{3}

\setlength{\parindent}{0em}
\topmargin -1.0cm
\oddsidemargin 0cm
\evensidemargin 0cm
\setlength{\textheight}{9.2in}
\setlength{\textwidth}{6.0in}

\newcommand{\Aufgabe}[1]{
        {
        \vspace*{0.5cm}
        \textbf{Hausaufgabe #1}
        \vspace*{0.2cm}
    }
}

\hypersetup{
    pdftitle = {\Fach{}: Übungsblatt \Uebungsblatt{}},
    pdfauthor = {\Name},
    pdfborder = {0 0 0}
}

\lstset{
    language=java,
    basicstyle=\footnotesize\tt,
    showtabs=false,
    tabsize=2,
    captionpos=b,
    breaklines=true,
    extendedchars=true,
    showstringspaces=false,
    flexiblecolumns=true,
}

\title{Übungsblatt \Uebungsblatt{}}
\author{\Name{}}

\begin{document}
    \thispagestyle{fancy}
    \lhead{\Fach{} \\ \small \Name{} - \Matrikelnummer{}}
    \rhead{\Semester{} \\  Übungsgruppe \Seminargruppe{}}
    \vspace*{0.2cm}
    \begin{center}
        \LARGE \textbf{Übungsblatt \Uebungsblatt{}}
    \end{center}
    \vspace*{0.2cm}

    \Aufgabe{5} \\
    \begin{enumerate}[(a)]
        \item
        $ r_1 = a^* + b^* + ((a + b)^* \cdot (a + bb))^* $ \\ \\
        Ein Wort aus dem Alphabet $ \{a, b\}^* $, welches nicht das Suffix $ ab $ hat kann neben den Basisf"allen
        von $ \epsilon, a, b $ entweder mit $ ba, bb $ oder $ aa $ enden.
        Der angegebene Ausdruck deckt also einmal diese Basisf"alle, und im Fall eines l"angeren Wortes mit abwechselnden
        $ a $s und $ b $s (also $ (a + b)^* $) stellt er sicher, dass das Wort entweder mit $ a $ oder $ bb $ terminiert,
        sodass der Suffix $ ab $ nicht zustande kommen kann.

        \item
        $ r_2 = b^* \cdot (ab^* ab^*)^* + a^* b \cdot (a^* b a^* b a^*) $ \\ \\
        Im ersten Fall ($ |w|_a $ gerade) k"onnen wir an jeder Stelle des Wortes
        beliebig viele $ b $s lesen, allerdings muss die Menge an $ a $s immer gerade bleiben,
        also ist im umklammerten Ausdruck eine gerade Menge an $ a $s enthalten,
        welche beliebig oft mit beliebig vielen $ b $s konkateniert werden k"onnen.
        Im zweiten Fall ($ |w|_b $ ungerade) muss mindestens ein $ b $ gelesen werden, allerdings k"onnen davor und dazwischen
        wieder beliebige $ a $s gelesen werden, und es soll die M"oglichkeit geben nach dem ersten $ b $ weiter eine
        gerade Menge an $ b $s zu schreiben, welche durch das erste $ b $ immer eine ungerade Anzahl ergeben werden.
    \end{enumerate}

    \Aufgabe{6} \\
    $ \mathcal{A} $ \\
    \begin{tikzpicture} [node distance = 3cm, on grid]
        \node (q0) [state, initial, initial text = {}] {$q_0$};
        \node (q1) [state, accepting, right = of q0] {$q_1$};

        \path [-stealth]
        (q0) edge [above, bend left] node {$a$} (q1)
        (q1) edge [loop above] node {$b$} ()
        (q1) edge [below, bend left] node {$\epsilon$} (q0)
        (q0) edge [below, bend right = 80] node {$\epsilon$} (q1)
    \end{tikzpicture}

\end{document}