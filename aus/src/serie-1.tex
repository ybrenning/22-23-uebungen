\documentclass[a4paper,12pt]{article}
\usepackage{fancyhdr}
\usepackage{fancyheadings}
\usepackage[ngerman,german]{babel}
\usepackage{german}
\usepackage[utf8]{inputenc}
%\usepackage[latin1]{inputenc}
\usepackage[active]{srcltx}
\usepackage{algorithm}
\usepackage[noend]{algorithmic}
\usepackage{amsmath}
\usepackage{amssymb}
\usepackage{amsthm}
\usepackage{bbm}
\usepackage{enumerate}
\usepackage{graphicx}
\usepackage{ifthen}
\usepackage{listings}
\usepackage{struktex}
\usepackage{hyperref}
\usepackage[T1]{fontenc}
\usepackage{amsfonts}
\usepackage{tikz}
\usetikzlibrary{automata, arrows.meta, positioning, calc}

%%%%%%%%%%%%%%%%%%%%%%%%%%%%%%%%%%%%%%%%%%%%%%%%%%%%%%
%%%%%%%%%%%%%%%%%%%%%%%%%%%%%%%%%%%%%%%%%%%%%%%%%%%%%%
\newcommand{\Fach}{Automaten und Sprachen}
\newcommand{\Name}{Yannick Brenning, Jean R"other}
\newcommand{\Seminargruppe}{C}
\newcommand{\Matrikelnummer}{3732848, 3796826}
\newcommand{\Semester}{WiSe 22/23}
\newcommand{\Uebungsblatt}{1} %  <-- UPDATE ME
%%%%%%%%%%%%%%%%%%%%%%%%%%%%%%%%%%%%%%%%%%%%%%%%%%%%%%
%%%%%%%%%%%%%%%%%%%%%%%%%%%%%%%%%%%%%%%%%%%%%%%%%%%%%%

\setlength{\parindent}{0em}
\topmargin -1.0cm
\oddsidemargin 0cm
\evensidemargin 0cm
\setlength{\textheight}{9.2in}
\setlength{\textwidth}{6.0in}

\newcommand{\Aufgabe}[1]{
        {
        \vspace*{0.5cm}
        \textbf{Hausaufgabe #1}
        \vspace*{0.2cm}
    }
}

\hypersetup{
    pdftitle = {\Fach{}: Übungsblatt \Uebungsblatt{}},
    pdfauthor = {\Name},
    pdfborder = {0 0 0}
}

\lstset{
    language=java,
    basicstyle=\footnotesize\tt,
    showtabs=false,
    tabsize=2,
    captionpos=b,
    breaklines=true,
    extendedchars=true,
    showstringspaces=false,
    flexiblecolumns=true,
}

\title{Übungsblatt \Uebungsblatt{}}
\author{\Name{}}

\begin{document}
    \thispagestyle{fancy}
    \lhead{\Fach{} \\ \small \Name{} - \Matrikelnummer{}}
    \rhead{\Semester{} \\  Übungsgruppe \Seminargruppe{}}
    \vspace*{0.2cm}
    \begin{center}
        \LARGE \textbf{Übungsblatt \Uebungsblatt{}}
    \end{center}
    \vspace*{0.2cm}

    \Aufgabe{4} \\
        \begin{enumerate}[(a)]
            \item
            $ b \in L(\mathcal{A}_1) $ \\
            $\epsilon \in L(\mathcal{A}_2) $
            \item
            $ bc \notin L(\mathcal{A}_1) $ \\
            $ aa \notin L(\mathcal{A}_2) $
            \item
            $ L(\mathcal{A}_1) = \{a^* \cdot b \cdot \{a, b\}^*\} $
            $ L(\mathcal{A}_2)  = \{w \in \{a, b\}^* \; | \; |w|_b \leq |w|_a \leq 4\} $
        \end{enumerate}

%    \Aufgabe{5} \\
%    \begin{tikzpicture}[node distance = 2.5cm, on grid, auto]
%        \node (1) [state, initial] {$q_0$};
%        \node (2) [state, above right of = 1] {$q_1$};
%        \node (3) [state, accepting, under right of = 2] {$q_3$};

%    \path [-stealth]
%    \end{tikzpicture} \leavevmode \\[2\baselineskip]
%    (1) edge [above] node {$b$} (2)
%    (2) edge [above] node {$a,b$} (3)
%    (3) edge [above] node {$a,b$} (4)
    \Aufgabe{5} \\
    $ \mathcal{A}_1 $ \\
    \begin{tikzpicture}[node distance = 2cm, on grid]
        % \draw [help lines] (-1, 1) grid (3, -1);
        \node (q0) [state, initial, accepting, initial text = {}] {$q_0$};
        \node (q1) [state, accepting, right = of q0] {$q_1$};
        \node (q2) [state, right = of q1] {$q_2$};

        \path [-stealth]
        (q0) edge [loop above] node {$a$} ()
        (q0) edge [above] node {$b$} (q1)
        (q1) edge [loop above] node {$a$} ()
        (q1) edge [above] node {$b$} (q2)
        (q2) edge [loop above] node {$a, b$} ()
    \end{tikzpicture}

    $ \mathcal{A}_2 $ \\
    \begin{tikzpicture}[node distance = 2cm, on grid]
        % \draw [help lines] (-1, 1) grid (3, -1);
        \node (q0) [state, initial, initial text = {}] {$q_0$};
        \node (q1) [state, accepting, right = of q0] {$q_1$};

        \path [-stealth]
        (q0) edge [loop above] node {$b$} ()
        (q0) edge [bend left, above] node {$a$} (q1)
        (q1) edge [loop above] node {$a$} ()
        (q1) edge [bend left, below] node {$b$} (q0)
    \end{tikzpicture}

    $ \mathcal{A}_3 $ \\
    \begin{tikzpicture}[node distance = 2.5cm, on grid]
        % \draw [help lines] (-1, 1) grid (3, -1);
        \node (q0) [state, initial, initial text = {}] {$q_0$};
        \node (q1) [state, right = of q0] {$q_1$};
        \node (q2) [state, right = of q1] {$q_2$};
        \node (q3) [state, accepting, right = of q2] {$q_3$};

        \path [-stealth]
        (q0) edge [above] node {$a, b, c$} (q1)
        (q1) edge [above] node {$a, b, c$} (q2)
        (q2) edge [above] node {$a, b, c$} (q3)
        (q3) edge [bend left, below] node {$a, b, c$} (q1)
    \end{tikzpicture}

    $ \mathcal{A}_4 $ \newpage

    $ \mathcal{A}_5 $ \\
    \begin{tikzpicture}[node distance = 2cm, on grid]
        % \draw [help lines] (-1, 1) grid (3, -1);
        \node (q0) [state, initial, initial text = {}] {$q_0$};
        \node (q1) [state, right = of q0] {$q_1$};
        \node (q2) [state, below = of q1] {$q_2$};
        \node (q3) [state, accepting, right = of q1] {$q_3$};

        \path [-stealth]
        (q0) edge [loop above] node {$a$} ()
        (q0) edge [above] node {$b$} (q1)
        (q1) edge [loop above] node {$a$} ()
        (q1) edge [above] node {$b$} (q3)
        (q3) edge [loop above] node {$a$} ()
        (q3) edge [below right] node {$b$} (q2)
        (q1) edge [left] node {$b$} (q2)
        (q2) edge [loop below] node {$a, b$} ()
    \end{tikzpicture}

    \Aufgabe{6} \\
    \begin{enumerate}[(a)]
        \item
        $L_1 \cdot (L_2 \cup L_3) \Leftrightarrow \{u \cdot v \; | \; u \in L_1 \land v \in L_2 \text{ oder } L_3\}$
        \begin{flalign*}
            (L_1 \cdot L_2) \cup (L_1 \cdot L_3) & \Leftrightarrow
            \{u \cdot v \; | \; u \in L_1 \land v \in L_2\}
            \cup \{u \cdot v \; | \; u \in L_1 \land v \in L_3\} && \\
            & \Leftrightarrow
            \{u \cdot v \; | \; u \in L_1 \land v \in L_2 \text{ oder } L_3\} && \\
            & \Leftrightarrow
            L_1 \cdot (L_2 \cup L_3)
        \end{flalign*}
        \item
        \begin{flalign*}
            L_1^* \cdot L_1 & = \bigcup_{i\in\mathbb N_0} L^i \cdot L_1 && \\
            \text{Sei } L_1 & = \{a, b\}. \text{ Dann gilt } L_1^* = \{\epsilon, a, b, aa, bb, ab, ba, aaa, \mathellipsis\}
        \end{flalign*}
    \end{enumerate}
    
    \Aufgabe{7}
    \begin{enumerate}[(a)]
        \item Der Automat $ \mathcal{A}  = \{\{q_0\}, \{a\}, \delta, \{q_0\}\}, \delta(q_0, a) = q_0 $ hat als einzigen
        akzeptierenden Zustand den Anfangszustand $q_0$, allerdings ist in diesem Fall $L(\mathcal{A}) = \{a\}^*$

        \item Ein deterministisch endlicher Automat $ \mathcal{A} $ hat f"ur jede m"ogliche Kombination von
        Zustand und Symbol einen Folgezustand definiert, und da alle Zust"ande akzeptierend sind, akzeptiert $ \mathcal{A} $
        auch alle W"orter "uber dem Alphabet $ \Sigma $, also $ \Sigma^* $.
    \end{enumerate}

\end{document}