\documentclass[a4paper,12pt]{article}
\usepackage{fancyhdr}
\usepackage{fancyheadings}
\usepackage[ngerman,german]{babel}
\usepackage{german}
\usepackage[utf8]{inputenc}
%\usepackage[latin1]{inputenc}
\usepackage[active]{srcltx}
\usepackage{algorithm}
\usepackage[noend]{algorithmic}
\usepackage{amsmath}
\usepackage{amssymb}
\usepackage{amsthm}
\usepackage{bbm}
\usepackage{enumerate}
\usepackage{graphicx}
\usepackage{ifthen}
\usepackage{listings}
\usepackage{struktex}
\usepackage{hyperref}
\usepackage[T1]{fontenc}
\usepackage{amsfonts}
\usepackage{tikz}
\usetikzlibrary{automata, arrows.meta, positioning, calc}

%%%%%%%%%%%%%%%%%%%%%%%%%%%%%%%%%%%%%%%%%%%%%%%%%%%%%%
%%%%%%%%%%%%%%%%%%%%%%%%%%%%%%%%%%%%%%%%%%%%%%%%%%%%%%
\newcommand{\Fach}{Automaten und Sprachen}
\newcommand{\Name}{Yannick Brenning, Jean R"other}
\newcommand{\Seminargruppe}{C}
\newcommand{\Matrikelnummer}{3732848, 3796826}
\newcommand{\Semester}{WiSe 22/23}
\newcommand{\Uebungsblatt}{1} %  <-- UPDATE ME
%%%%%%%%%%%%%%%%%%%%%%%%%%%%%%%%%%%%%%%%%%%%%%%%%%%%%%
%%%%%%%%%%%%%%%%%%%%%%%%%%%%%%%%%%%%%%%%%%%%%%%%%%%%%%

\setlength{\parindent}{0em}
\topmargin -1.0cm
\oddsidemargin 0cm
\evensidemargin 0cm
\setlength{\textheight}{9.2in}
\setlength{\textwidth}{6.0in}

\newcommand{\Aufgabe}[1]{
        {
        \vspace*{0.5cm}
        \textbf{Hausaufgabe #1}
        \vspace*{0.2cm}
    }
}

\hypersetup{
    pdftitle = {\Fach{}: Übungsblatt \Uebungsblatt{}},
    pdfauthor = {\Name},
    pdfborder = {0 0 0}
}

\lstset{
    language=java,
    basicstyle=\footnotesize\tt,
    showtabs=false,
    tabsize=2,
    captionpos=b,
    breaklines=true,
    extendedchars=true,
    showstringspaces=false,
    flexiblecolumns=true,
}

\title{Übungsblatt \Uebungsblatt{}}
\author{\Name{}}

\begin{document}
    \thispagestyle{fancy}
    \lhead{\Fach{} \\ \small \Name{} - \Matrikelnummer{}}
    \rhead{\Semester{} \\  Übungsgruppe \Seminargruppe{}}
    \vspace*{0.2cm}
    \begin{center}
        \LARGE \textbf{Übungsblatt \Uebungsblatt{}}
    \end{center}
    \vspace*{0.2cm}

    \Aufgabe{4}
        \begin{enumerate}[(a)]
            \item
            $ b \in L(\mathcal{A}_1) $ \\
            $\epsilon \in L(\mathcal{A}_2) $

            \item
            $ bc \notin L(\mathcal{A}_1) $ \\
            $ aa \notin L(\mathcal{A}_2) $

            \item
            $ L(\mathcal{A}_1) = \{a^* \cdot b \cdot \{a, b\}^*\} $ \\
            $ L(\mathcal{A}_2)  = \{w \in \{a, b\}^* \setminus \{aa, bb\} \; | \; |w| \mod 2 = 0\} $
        \end{enumerate}

    \Aufgabe{5} \\
    $ \mathcal{A}_1 $ \\
    \begin{tikzpicture}[node distance = 2cm, on grid]
        \node (q0) [state, initial, accepting, initial text = {}] {$q_0$};
        \node (q1) [state, accepting, right = of q0] {$q_1$};
        \node (q2) [state, right = of q1] {$q_2$};

        \path [-stealth]
        (q0) edge [loop above] node {$a$} ()
        (q0) edge [above] node {$b$} (q1)
        (q1) edge [loop above] node {$a$} ()
        (q1) edge [above] node {$b$} (q2)
        (q2) edge [loop above] node {$a, b$} ()
    \end{tikzpicture}

    $ \mathcal{A}_2 $ \\
    \begin{tikzpicture}[node distance = 2cm, on grid]
        \node (q0) [state, initial, initial text = {}] {$q_0$};
        \node (q1) [state, accepting, right = of q0] {$q_1$};

        \path [-stealth]
        (q0) edge [loop above] node {$b$} ()
        (q0) edge [bend left, above] node {$a$} (q1)
        (q1) edge [loop above] node {$a$} ()
        (q1) edge [bend left, below] node {$b$} (q0)
    \end{tikzpicture}

    $ \mathcal{A}_3 $ \\
    \begin{tikzpicture}[node distance = 2.5cm, on grid]
        \node (q0) [state, initial, initial text = {}] {$q_0$};
        \node (q1) [state, right = of q0] {$q_1$};
        \node (q2) [state, right = of q1] {$q_2$};
        \node (q3) [state, accepting, right = of q2] {$q_3$};

        \path [-stealth]
        (q0) edge [above] node {$a, b, c$} (q1)
        (q1) edge [above] node {$a, b, c$} (q2)
        (q2) edge [above] node {$a, b, c$} (q3)
        (q3) edge [bend left, below] node {$a, b, c$} (q1)
    \end{tikzpicture}

    $ \mathcal{A}_4 $ \\
    \begin{tikzpicture}[node distance = 2.5cm, on grid]
        \node (q0) [state, initial, initial text = {}] {$q_0$};
        \node (q1) [state, right = of q0] {$q_1$};
        \node (q2) [state, accepting, right = of q1] {$q_2$};
        \node (q3) [state, accepting, below = of q2] {$q_3$};
        \node (q4) [state, accepting, left = of q3] {$q_4$};
        \node (q5) [state, left = of q4] {$q_5$};

        \path [-stealth]
        (q0) edge [above] node {$a, b, c$} (q1)
        (q1) edge [above] node {$a, b, c$} (q2)
        (q2) edge [right] node {$a, b, c$} (q3)
        (q3) edge [above] node {$a, b, c$} (q4)
        (q4) edge [above] node {$a, b, c$} (q5)
        (q5) edge [left] node {$a, b, c$} (q0)
    \end{tikzpicture}

    $ \mathcal{A}_5 $ \\
    \begin{tikzpicture}[node distance = 2cm, on grid]
        \node (q0) [state, initial, initial text = {}] {$q_0$};
        \node (q1) [state, right = of q0] {$q_1$};
        \node (q2) [state, below = of q1] {$q_2$};
        \node (q3) [state, accepting, right = of q1] {$q_3$};

        \path [-stealth]
        (q0) edge [loop above] node {$a$} ()
        (q0) edge [above] node {$b$} (q1)
        (q1) edge [loop above] node {$a$} ()
        (q1) edge [above] node {$b$} (q3)
        (q3) edge [loop above] node {$a$} ()
        (q3) edge [below right] node {$b$} (q2)
        (q1) edge [left] node {$b$} (q2)
        (q2) edge [loop below] node {$a, b$} ()
    \end{tikzpicture}

    \Aufgabe{6}
    \begin{enumerate}[(a)]
        \item
        \begin{flalign*}
            L_1 \cdot (L_2 \cup L_3) & \Leftrightarrow \{u \cdot v \; | \; u \in L_1 \land v \in L_2 \text{ oder } L_3\} && \\ \\
            (L_1 \cdot L_2) \cup (L_1 \cdot L_3) & \Leftrightarrow
            \{u \cdot v \; | \; u \in L_1 \land v \in L_2\}
            \cup \{u \cdot v \; | \; u \in L_1 \land v \in L_3\} && \\
            & \Leftrightarrow
            \{u \cdot v \; | \; u \in L_1 \land v \in L_2 \text{ oder } L_3\} && \\
            & \Leftrightarrow
            L_1 \cdot (L_2 \cup L_3)
        \end{flalign*}

        \item
        \begin{flalign*}
            L_1^* & = \bigcup_{i\in\mathbb N_0} L^i && \\
            L_1^* \cdot L_1 & = \bigcup_{i\in\mathbb N_0} L^i \cdot L_1
        \end{flalign*}
        Widerlegung durch Gegenbeispiel \\
        Sei $L_1 & = \{a, b\}$.
        Dann gilt: \\
        \begin{flalign*}
            L_1^* & = \{\epsilon, a, b, aa, bb, ab, ba, aaa, \mathellipsis\}. && \\
            L_1^* \cdot L_1 & = \{a, b, aa, ab, ba, bb, aba, abb, baa, bab, \mathellipsis\}
        \end{flalign*}

        Aufgrund der Konkatenation mit $ L_1 $ enth"alt die Sprache $ L_1^* \cdot L_1 $ nicht mehr das leere Wort.
        Da also $ \epsilon \in L_1^*, \epsilon \notin L_1^* \cdot L_1 $, gilt dementsprechend $ L_1^* \cdot L_1 \neq L_1^* $.

        \newpage
        \item
        \begin{flalign*}
            \text{Zwei Sprachen } L_4, L_5 \text{ sind gleich, wenn gilt: } \forall w \in L_4 \Leftrightarrow w \in L_5 && \\
        \end{flalign*}

        $ L_4 := \overline{\{a\} \cdot \{b\} \cdot \{a\}} $ \\
        $ L_5 := \{b\}^* \cdot \{a\} \cdot \{b\}^* $ \\ \\

        Widerlegung durch Gegenbeispiel ($ \exists w \in L_4, w \notin L_5 $): \\
        $ \epsilon \in L_4, \epsilon \notin L_5 $
        $ \Rightarrow \overline{\{a\} \cdot \{b\} \cdot \{a\}} \neq \{b\}^* \cdot \{a\} \cdot \{b\}^* $
    \end{enumerate}

    \Aufgabe{7}
    \begin{enumerate}[(a)]
        \item
        Der Automat $ \mathcal{A}  = \{\{q_0\}, \{a\}, \delta, \{q_0\}\}, \delta(q_0, a) = q_0 $ hat als einzigen
        akzeptierenden Zustand den Anfangszustand $ q_0 $, allerdings ist in diesem Fall $ L(\mathcal{A}) = \{a\}^* $.

        \item
        Ein deterministisch endlicher Automat $ \mathcal{A} $ hat f"ur jede m"ogliche Kombination von
        Zustand und Symbol einen Folgezustand definiert, und da alle Zust"ande akzeptierend sind, akzeptiert $ \mathcal{A} $
        auch alle W"orter "uber dem Alphabet $ \Sigma $, also $ \Sigma^* $.

        \item
        Ein Automat mit zwei akzeptierenden Zust"anden hat trivialerweise zwei M"oglichkeiten, einen Lauf zu beenden,
        und da DEAs keine $ \epsilon $-"Uberg"ange haben, muss bei jedem Zustands"ubergang ein nichtleeres Symbol gelesen werden, wodurch
        $ L(\mathcal{A}) $ trivialerweise mindestens zwei W"orter enthalten muss (im minimalen Fall von nur einem
        Zustands"ubergang w"aren diese das leere Wort und des Symbol des "Ubergangs).

        \item
        Da die Sprache $ L = \{a\}^* \cdot \{b\}^* $ das leere Wort enth"alt, muss der Anfangszustand auch akzeptierend
        sein, und es muss einen zweiten akzeptierenden Zustand mit den "Uberg"angen $ \delta(q_0, b) = q_1 , \delta(q_1, b) = q_1 $
        geben, um die Konkatenation mit $ \{b\}^* $ akzeptieren zu k"onnen.
        Es existiert also eine Sprache $ L $, die nur von DEAs mit mindestens zwei akzeptierenden Zust"anden erkannt wird.
    \end{enumerate}

\end{document}