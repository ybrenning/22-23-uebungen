\documentclass[a4paper,12pt]{article}
\usepackage{fancyhdr}
\usepackage{fancyheadings}
\usepackage[ngerman,german]{babel}
\usepackage{german}
\usepackage[utf8]{inputenc}
%\usepackage[latin1]{inputenc}
\usepackage[active]{srcltx}
\usepackage{algorithm}
\usepackage[noend]{algorithmic}
\usepackage{amsmath}
\usepackage{amssymb}
\usepackage{amsthm}
\usepackage{bbm}
\usepackage{enumerate}
\usepackage{graphicx}
\usepackage{ifthen}
\usepackage{listings}
\usepackage{struktex}
\usepackage{hyperref}
\usepackage[T1]{fontenc}
\usepackage{amsfonts}
\usepackage{tikz}
\usetikzlibrary{automata, arrows.meta, positioning, calc}

%%%%%%%%%%%%%%%%%%%%%%%%%%%%%%%%%%%%%%%%%%%%%%%%%%%%%%
%%%%%%%%%%%%%%%%%%%%%%%%%%%%%%%%%%%%%%%%%%%%%%%%%%%%%%
\newcommand{\Fach}{Automaten und Sprachen}
\newcommand{\Name}{Yannick Brenning}
\newcommand{\Seminargruppe}{C}
\newcommand{\Matrikelnummer}{3732848}
\newcommand{\Semester}{WiSe 21/22}
\newcommand{\Uebungsblatt}{1} %  <-- UPDATE ME
%%%%%%%%%%%%%%%%%%%%%%%%%%%%%%%%%%%%%%%%%%%%%%%%%%%%%%
%%%%%%%%%%%%%%%%%%%%%%%%%%%%%%%%%%%%%%%%%%%%%%%%%%%%%%

\setlength{\parindent}{0em}
\topmargin -1.0cm
\oddsidemargin 0cm
\evensidemargin 0cm
\setlength{\textheight}{9.2in}
\setlength{\textwidth}{6.0in}

\newcommand{\Aufgabe}[1]{
        {
        \vspace*{0.5cm}
        \textbf{Hausaufgabe 1}
        \vspace*{0.2cm}
    }
}

\hypersetup{
    pdftitle = {\Fach{}: Übungsblatt \Uebungsblatt{}},
    pdfauthor = {\Name},
    pdfborder = {0 0 0}
}

\lstset{
    language=java,
    basicstyle=\footnotesize\tt,
    showtabs=false,
    tabsize=2,
    captionpos=b,
    breaklines=true,
    extendedchars=true,
    showstringspaces=false,
    flexiblecolumns=true,
}

\title{Übungsblatt \Uebungsblatt{}}
\author{\Name{}}

\begin{document}
    \thispagestyle{fancy}
    \lhead{\Fach{} \\ \small \Name{} - \Matrikelnummer{}}
    \rhead{\Semester{} \\  Übungsgruppe \Seminargruppe{}}
    \vspace*{0.2cm}
    \begin{center}
        \LARGE \textbf{Übungsblatt \Uebungsblatt{}}
    \end{center}
    \vspace*{0.2cm}

    \Aufgabe{4}
        \begin{enumerate}[(a)]
            \item
            $ b \in L(\mathcal{A}_1) $ \\
            $\epsilon \in L(\mathcal{A}_2) $
            \item
            $ bc \notin L(\mathcal{A}_1) $ \\
            $ c \notin L(\mathcal{A}_2) $
            \item
            $ L(\mathcal{A}_1) = \{a^* \cdot b \cdot \{a, b\}^*\} $
            $ L(\mathcal{A}_2)  = \{w \in \{a, b\}^* \; | \; |w|_b \leq |w|_a \leq 4\} $
        \end{enumerate}

%    \Aufgabe{5} \\
%    \begin{tikzpicture}[node distance = 2.5cm, on grid, auto]
%        \node (1) [state, initial] {$q_0$};
%        \node (2) [state, above right of = 1] {$q_1$};
%        \node (3) [state, accepting, under right of = 2] {$q_3$};

%    \path [-stealth]
%    \end{tikzpicture} \leavevmode \\[2\baselineskip]
%    (1) edge [above] node {$b$} (2)
%    (2) edge [above] node {$a,b$} (3)
%    (3) edge [above] node {$a,b$} (4)

    \Aufgabe{6} \\
    \begin{enumerate}[(a)]
        \item
        $L_1 \cdot (L_2 \cup L_3) \Leftrightarrow \{u \cdot v \; | \; u \in L_1 \land v \in L_2 \text{ oder } L_3\}$
        \begin{flalign*}
            (L_1 \cdot L_2) \cup (L_1 \cdot L_3) & \Leftrightarrow
            \{u \cdot v \; | \; u \in L_1 \land v \in L_2\}
            \cup \{u \cdot v \; | \; u \in L_1 \land v \in L_3\} && \\
            & \Leftrightarrow
            \{u \cdot v \; | \; u \in L_1 \land v \in L_2 \text{ oder } L_3\} && \\
            & \Leftrightarrow
            L_1 \cdot (L_2 \cup L_3)
        \end{flalign*}
        \item
        \begin{flalign*}
            L_1^* \cdot L_1 & = \bigcup_{i\in\mathbb N_0} L^i \cdot L_1 && \\
            \text{Sei } L_1 & = \{a, b\}. \text{ Dann gilt } L_1^* = \{\epsilon, a, b, aa, bb, ab, ba, aaa, \mathellipsis\}
        \end{flalign*}
    \end{enumerate}

\end{document}